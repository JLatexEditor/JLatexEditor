\documentclass[10pt,handout]{beamer}

\usepackage[english]{babel}
\usepackage[latin1]{inputenc}
\usepackage{comment}
%\usepackage{proof}
\usepackage{graphicx}
%\usepackage{rotate}

\mode<presentation>
{
  \usetheme{Warsaw}  %% \usetheme{default} 
  \setbeamercovered{transparent}
  \usecolortheme{seahorse}
  \usecolortheme{rose} %% \usecolortheme{crane} % (oranje)
  \setbeamertemplate{navigation symbols}{}
\setbeamertemplate{itemize items}[triangle]
\setbeamertemplate{sections/subsections in toc}[sections numbered]
\setbeamercolor{math text}{fg=darkgreen}
}


\input{wsg.def}


%------------------------------------------------------------------------


%%\usepackage[german]{babel}
% oder was auch immer

\usepackage[latin1]{inputenc}
% oder was auch immer

\usepackage{times}
\usepackage[T1]{fontenc}
% Oder was auch immer. Zu beachten ist, das Font und Encoding passen
% m�ssen. Falls T1 nicht funktioniert, kann man versuchen, die Zeile
% mit fontenc zu l�schen.



\title[Watching Streams Grow]{\scalebox{0.99}%
         {\hspace*{-1.5pt}Watching \hspace*{-1.5pt}Streams \hspace*{-1.5pt}Grow: 
          \hspace*{-1.5pt}The \hspace*{-1.5pt}Pebbleflow \hspace*{-1.5pt}Method}}

\author[Endrullis, Grabmayer, Hendriks, Isihara, Klop]{
        %% [\magenta{Fanta~5}]
    %% %{\magenta{Fanta~5:}}\\
  J\"{o}rg Endrullis         %\inst{1}
  \and {Clemens Grabmayer}     %\inst{2}
  \and Dimitri Hendriks      \\%\inst{1}\\
  \and Ariya Isihara         %\inst{1}
  \and Jan Willem Klop       %\inst{1}\\
}

\institute{
  Universiteit Utrecht, 
  Vrije Universiteit
  %%                               \\[2ex]
  %% {\chocolate{\large NWO-Bricks Project Infinity}}  %{{ $\boldsymbol{\infty}$}}
           }
  
%% \institute{
%%   Department of Philosophy, \emph{Universiteit Utrecht}\\[0.5ex]
%%   Department of Computer Science, \emph{Vrije Universiteit}\\[0.5ex]
%%   Department of Computer Science, \emph{Radboud Universiteit}   
%%            }


\date{
  {NWO Projects \chocolate{Infinity}
                and \chocolate{ProvCyc}}\\[2ex]
  \forestgreen{TF-lunch}, \emph{UU}\\ 
      6 November 2007}


\subject{Informatik}
% Dies wird lediglich in den PDF Informationskatalog einf�gt. Kann gut
% weggelassen werden.


% Falls eine Logodatei namens "university-logo-filename.xxx" vorhanden
% ist, wobei xxx ein von latex bzw. pdflatex lesbares Graphikformat
% ist, so kann man wie folgt ein Logo einf�gen:

% \pgfdeclareimage[height=0.5cm]{university-logo}{university-logo-filename}
% \logo{\pgfuseimage{university-logo}}


%% % Folgendes sollte gel�scht werden, wenn man nicht am Anfang jedes
%% % Unterabschnitts die Gliederung nochmal sehen m�chte.
   \AtBeginSection[]
   {
   \begin{frame}<beamer>
     \frametitle{Overview}
     \tableofcontents[currentsection] %[...,currentsubsection]
   \end{frame}
    }


% Falls Aufz�hlungen immer schrittweise gezeigt werden sollen, kann
% folgendes Kommando benutzt werden:

%\beamerdefaultoverlayspecification{<+->}


%----------------------------------------------------------------------------------




\begin{comment}
  %
  %abstract
  %
  We describe an algorithm for deciding productivity of a large and
  natural class of recursive stream definitions.
  A stream definition is called  `productive' if  
  it can be evaluated continuously
  in such a way that a uniquely determined stream is obtained as the limit.
  Whereas productivity is undecidable for stream definitions in general,
  we show that it can be decided for `pure' stream definitions.
  For every pure stream definition the process of its evaluation
  can be modelled by the dataflow of abstract stream elements,
  called `pebbles', in a finite `pebbleflow net(work)'.
  And the production of a pebbleflow net
  associated with a pure stream definition, that is,
  the amount of pebbles the net is able to produce at its output port,
  can be calculated by reducing nets to trivial nets.
  %
\end{comment}

%----------------------------------------------------------------

\begin{document}

%\renewcommand{\pause}{}

\begin{frame}
	\titlepage
\end{frame}

\begin{frame}
  \frametitle{Outline}
  \tableofcontents[pausesections]
\end{frame}

%------------------------------------------------------------------------
%------------------------------------------------------------------------
\section{Introduction}
%------------------------------------------------------------------------
%------------------------------------------------------------------------


%------------------------------------------------------------------------
\subsection{Productivity}
%------------------------------------------------------------------------

%% \begin{frame}%%%%%%%%%%%%%%%%%%%%%%%%%%%%%%%%%%%%%%%%%%%%%%%%%%%%%%%%%%%%
%%   \frametitle{Productivity}
%% 
%%   \begin{itemize}[<+->]
%%   \item
%%     %Under which conditions can a finite expression
%%     %be said to represent an infinite object?
%%     When does a \emph{finite expression}
%%     represent an \emph{infinite object}?
%%   \item
%%     When do we accept an infinite object 
%%     defined in terms of itself?
%%   \item
%%     Answer: if its definition is \emph{productive}:\\
%%     if it reduces to a \emph{constructor normal form}.
%%     %it can generate leading constructors to an arbitrary depth
%%   %\item
%%     %Includes \emph{lazy} objects: not necessarily well-founded 
%%     %trees built out of constructors.
%%   \item
%%     A productive \emph{process} continuously turns input into output.
%%   \item
%%     \emph{Undecidable}, in general.
%%   \item
%%     Examples: processes, \emph{streams}, 
%%     infinite proofs, operating systems, \ldots.
%%   \end{itemize}
%% 
%% \end{frame}%%%%%%%%%%%%%%%%%%%%%%%%%%%%%%%%%%%%%%%%%%%%%%%%%%%%%%%%%%%%%% 
 
%------------------------------------------------------------------------
%\subsection{Problem}  
%------------------------------------------------------------------------

\begin{frame}
  \frametitle{The Problem} 

  \begin{itemize}%[<+->]
  \item
    When do we accept an infinite 
    object defined in terms of itself?
  \item
    When does a finite set of equations
    \emph{constructively} define
    a \emph{unique} infinite object?
  \item
    When is a recursive program \emph{productive}?  
  \vspace{2ex}
  \item
    If it evaluates to a unique infinite \emph{constructor normal form}
  \end{itemize}

  %\uncover<5->{
  \begin{example}
  \vspace{-2ex}
  \begin{align*}
    %\uncover<5->{
    \strcon{zeros} 
    &= \strcns{\datsymb{0}}{\strcon{zeros}}
    \\
    %}
    %\uncover<6->{
    \strcon{zeros} 
    &\red
    \strcns{\datsymb{0}}{\strcon{zeros}}
    \red
    \strcns{\datsymb{0}}{\strcns{\datsymb{0}}{\strcon{zeros}}}
    \red
    \cdots
    \red
    \strcns{\datsymb{0}}{\strcns{\datsymb{0}}{\strcns{\datsymb{0}}{\ldots}}}
    %} 
  \end{align*}
  \end{example}
  %}

  \begin{itemize}%[<+->]
  \item
    Other examples: %coinductive natural numbers,
    infinite trees, processes, coinductive proofs, %\emph{streams}, 
    \ldots
  \item
    In general, productivity is (highly) \emph{undecidable}
    ($\Pi^0_3$-complete).
    %
  \end{itemize}

\end{frame}  



%------------------------------------------------------------------------
%\subsection{Streams}
%------------------------------------------------------------------------


\begin{frame}%%%%%%%%%%%%%%%%%%%%%%%%%%%%%%%%%%%%%%%%%%%%%%%%%%%%%%%%%%%%
  \frametitle{Streams}

  \begin{itemize}%[<+->]
  \item
    The set $\str{A}$ of \emph{streams} (over set $A$)
    is defined by:
    \[
      \str{A} \defdby \{ \astr \where \astr \funin {\nat\to A} \}
    \]
    \vspace*{-1.5ex}
  \item
    $\str{A}$ is the greatest fixed point of: 
    \[\lambda{X}.\,A\times X\]
    %\[
    %  \str{A} \defdby A\times\str{A}
    %\]
    \vspace*{-1.5ex}
  \item
    `$\sstrcns$'\ is the \emph{stream constructor} symbol:
    \;\(\strcns{\datsymb{a}}{\astr}\)\;
    denotes the result of prepending 
    \;$\datsymb{a}\in A$\; to \;$\astr\in\str{A}$
  \item
    A recursive stream specification 
    \[
      \astrcon = \ldots\, {\astrcon} \,\ldots
    \]
    is \emph{productive} if the process of continually evaluating $\astrcon$ 
    \\
    results in an infinite constructor normal form:
    \[
      \astrcon
      \infred
      \strcns{\datsymb{a_0}}{\strcns{\datsymb{a_1}}{\strcns{\datsymb{a_2}}{\ldots}}}
    \]
  \end{itemize}
\end{frame}%%%%%%%%%%%%%%%%%%%%%%%%%%%%%%%%%%%%%%%%%%%%%%%%%%%%%%%%%%%%%%



%------------------------------------------------------------------------
%\subsection{Examples}
%------------------------------------------------------------------------


\begin{frame}%%%%%%%%%%%%%%%%%%%%%%%%%%%%%%%%%%%%%%%%%%%%%%%%%%%%%%%%%%%%
  \frametitle{Examples}
\begin{example}
  \begin{align*}
    \uncover<1>{
    \strread{\strcns{x}{\astr}} 
      &= \strcns{x}{\strread{\astr}} 
      & \\}
    \uncover<2>{
    \strfread{\strcns{x}{\strcns{y}{\astr}}}
      &= \strcns{x}{\strcns{y}{\strfread{\astr}}} 
      & \\}
    \uncover<1>{
    \strcon{fives} 
      &= \strcns{\datsymb{5}}{\strread{\strcon{fives}}} 
      & & \text{productive} \\}
    \uncover<2>{
    \strcon{fives'} 
      &= \strcns{\datsymb{5}}{\strfread{\strcon{fives'}}} 
      & & \text{not productive} \\}
    \uncover<3>{ 
    \strzipb{\strcns{x}{\astr}}{\strcns{y}{\bstr}} 
      &= \strcns{x}{\strcns{y}{\strzipb{\astr}{\bstr}}}
      & \\}
    \uncover<4>{
    \strzipa{\strcns{x}{\astr}}{\bstr} 
      &= \strcns{x}{\strzipa{\bstr}{\astr}} 
      & \\}
    \uncover<3>{
    \strcon{sevens} 
      &= \strcns{\datsymb{7}}{\strzipb{\strcon{sevens}}{\strtl{\strcon{sevens}}}}
      & & \text{not productive} \\} 
    \uncover<4>{
    \strcon{sevens'} 
      &= \strcns{\datsymb{7}}{\strzipa{\strcon{sevens'}}{\strtl{\strcon{sevens'}}}}
      & & \text{productive} }
  \end{align*}
\end{example}

\end{frame}%%%%%%%%%%%%%%%%%%%%%%%%%%%%%%%%%%%%%%%%%%%%%%%%%%%%%%%%%%%%%%




%------------------------------------------------------------------------
%------------------------------------------------------------------------
\section{Recursive Stream Specifications}
%------------------------------------------------------------------------
%------------------------------------------------------------------------


%------------------------------------------------------------------------
\subsection{Motivating Example}
%------------------------------------------------------------------------




%------------------------------------------------------------------------
%\subsection{Weakly guarded SFSs and Pure SCSs (Example)}
%------------------------------------------------------------------------

%------------------------------------------------------------------------
\subsection{Weakly Guarded Stream Function Specifications}
%------------------------------------------------------------------------

\begin{frame}%%%%%%%%%%%%%%%%%%%%%%%%%%%%%%%%%%%%%%%%%%%%%%%%%%%%%%%%%%%%
  \frametitle{Weakly Guarded Stream Function Specifications}

\begin{example}
  %
  \vspace*{2.5ex}
  %
\begin{center}
\vspace*{-0.5ex}
\scalebox{0.93}{
\renewcommand{\arraystretch}{1.25}
\begin{tabular}{|c|l|}
  \hline
  $\strtl{\strcns{x}{\astr}} \red \astr$ & \\[-0.0ex]
  $\strev{\strcns{x}{\astr}} \red \strcns{x}{\strod{\astr}}$ & \\[-0.0ex]
  $\strod{\strcns{x}{\astr}} \red \strev{\astr} $ 
  & SFS-layer\\[-0.0ex]
  $\strzip{\strcns{x}{\astr}}{\bstr} \red \strcns{x}{\strzip{\bstr}{\astr}}$
  & \\[-0.0ex]
  $ \stradd{ \strcns{x}{\astr} }{ \strcns{y}{\bstr} }
         \red   \strcns{\strad{x}{y}}
                           { \stradd{\astr}{\bstr} } $ & \\[-0.0ex]
  \hline
  $ \strad{x}{0} \red   x  $ \qquad\qquad
  $ \strad{x}{\strsucc{y}}   \red   \strsucc{\strad{x}{y}} $
  & data-layer \\
  \hline
\end{tabular}
               }
\end{center}
  %
  \mbox{}\\[0.5ex]
  %
\end{example}

\end{frame}%%%%%%%%%%%%%%%%%%%%%%%%%%%%%%%%%%%%%%%%%%%%%%%%%%%%%%%%%%%%%%
 
\begin{frame}%%%%%%%%%%%%%%%%%%%%%%%%%%%%%%%%%%%%%%%%%%%%%%%%%%%%%%%%%%%
  \frametitle{Weakly Guarded Stream Function Specifications}

\vspace*{-0.5ex}
%
\begin{example}[Continued] \label{ex:def:rsfds}
  %
  %% Consider again the SFS~$\atrs$ for 
  %% $\sstrev$, $\sstrod$, $\sstrzip$, $\sstrinv$, $\sstrtl$.
  %
  In an SFS we have `production cycles' like:
  %
  \begin{gather*}
    %
    \strev{\strcns{x}{\strcns{y}{\astr}}} 
        \red
      \strcns{x}{\strod{\strcns{y}{\astr}}}
        \red 
      \strcns{\chocolate{x}}{\strev{\astr}} 
                                                                   \\
    %\uncover<2->{
    \strod{\strcns{x}{\strcns{y}{\astr}}}
        \red
      \strev{\strcns{y}{\astr}}
        \red 
      \strcns{\chocolate{y}}{\strod{\astr}}
    %             }
                                                                   \\
    %\uncover<2->{
    \strzip{\strcns{x}{\astr}}{\strcns{y}{\bstr}} 
        \red 
      \strcns{x}{\strzip{\strcns{y}{\bstr}}{\astr}}
        \red
      \strcns{\chocolate{x}}{\strcns{\chocolate{y}}{\strzip{\astr}{\bstr}}}
    %             }
      %
  \end{gather*}                                              %\pause
  %
  We say that $\sstrev$, $\sstrod$, $\sstrzip$, and $\sstrinv$
  are \chocolate{weakly guarded}.\\[1.5ex]

  $\sstrtl$ is \emph{collapsing} in $\atrs$:
  %
  \begin{equation*}
    %
    \strtl{\strcns{x}{\astr}} \red \astr \; .
    %
  \end{equation*}
  %
\end{example}


\vspace*{0.5ex}
%

\end{frame}%%%%%%%%%%%%%%%%%%%%%%%%%%%%%%%%%%%%%%%%%%%%%%%%%%%%%%%%%%%%


\begin{frame}%%%%%%%%%%%%%%%%%%%%%%%%%%%%%%%%%%%%%%%%%%%%%%%%%%%%%%%%%%%%
  \frametitle{Weakly Guarded Stream Function Specifications}


\begin{definition}  
  %
  A TRS~%
  %
  $\atrs = \pair{\Sd\uplus\Ssf\uplus\{\sstrcns\}}{R_d\uplus R_{\mit{sf}}}$\\
  is a 
  \magenta{weakly guarded stream function specification (SFS)}
  if
  %
  \begin{enumerate}
    %
    \item $ \pair{\Sd}{R_d}$ is a sub-TRS,
        the \emph{data layer of $\atrs$}.%\pause
        %
    \item Each rule in $R_{\mit{sf}}$, the \emph{SFS-layer of $\atrs$},
      is of one of two forms:
      %
      \begin{align*}
        &
        \hspace*{-3ex}
        \funap{\astrfun}{
          (\strcns{\dm{x_{1,1}}}{\strcns{\ldots}{\strcns{\dm{x_{1,n_1}}}{\astr_1}}}),
          \ldots,
          (\strcns{\dm{x_{r,1}}}{\strcns{\ldots}{\strcns{\dm{x_{r_s,n_{r_s}}}}{\astr_{r_s}}}}),
          \vec{y}
                             } 
                               \\
        & \hspace*{-2ex}
        \red
        \strcns{\chocolate{t_{1}(\vec{x},\vec{y})}}{\strcns{\ldots}{\strcns{\chocolate{t_{m_{\hspace*{1pt}\astrfun}}(\vec{x},\vec{y})}}{
           \sigma_l }}} \; ,
                               \\
        &\hspace*{-2ex}
        \red 
        \strcns{\chocolate{t_{1}(\vec{x},\vec{y})}}{\strcns{\ldots}{\strcns{\chocolate{t_{m_{\hspace*{1pt}\astrfun}}(\vec{x},\vec{y})}}{
         \funap{\bstrfun}{\astr_{\funap{\pi_{\astrfun}}{1}},\ldots,\astr_{\funap{\pi_{\astrfun}}{r'_s}},
          t'_1(\vec{x},\vec{y}), \ldots, t'_{r'_d}(\vec{x},\vec{y})
                  }}}}  \; ,
        % 
      \end{align*}
      % 
      where
      $\pi_{\astrfun} \funin {\{1,\ldots,r'_s\} \to \{1,\ldots,r_s\}}$ 
      is injective
      in case $\astrfun\leadsto\bstrfun$.%\pause
   %%    % 
   %%    \begin{equation*}
   %%      %
   %%      \funap{\astrfun}{
   %%        (\strcns{t_{1,1}(\vec{x})}{\strcns{\ldots}{\strcns{t_{1,n_1}(\vec{x})}{\astr_1}}}),
   %%        \ldots,
   %%        (\strcns{t_{r,1}(\vec{x})}{\strcns{\ldots}{\strcns{t_{r,n_{r}}(\vec{x})}{\astr_{r}}}})
   %%                 }
  %%       \;\red\;   \btrm(\vec{x},\vec{\astr}) \; .
  %%       %
  %%     \end{equation*} 
  %%     %
      %
    \item \emph{Weakly guarded}:
      On every dependency cycle 
      $\astrfun \leadsto \bstrfun \leadsto \cdots \leadsto \astrfun$ 
      there is at least one \chocolate{guard}.
      
    %
  \end{enumerate}
  %
  %% (Signature parts $\asig_d\,$ for \emph{data symbols},
  %%  $\Ssf$ for \emph{stream function symbols};
  %%  rules $R_d$ for data symbols,
  %%  in $R_{\mit{sf}}$ for stream function symbols.)%\pause
  %
\end{definition}

%------------------------------------------------------------------------
\end{frame}%%%%%%%%%%%%%%%%%%%%%%%%%%%%%%%%%%%%%%%%%%%%%%%%%%%%%%%%%%%%%%


%------------------------------------------------------------------------
\subsection{Pure Stream Constant Specifications}
%------------------------------------------------------------------------


\begin{frame}%%%%%%%%%%%%%%%%%%%%%%%%%%%%%%%%%%%%%%%%%%%%%%%%%%%%%%%%%%%%
  \frametitle{Pure Stream Constant Specifications}

\begin{example}
  %
  \vspace*{2.5ex}
  %
\begin{center}
\scalebox{1.1}{
\renewcommand{\arraystretch}{1.1}
\begin{tabular}{|c|l|}
  \hline
  $\morse \red \strcns{\datsymb{0}}{\strzip{\strinv{\morse}}{\strtl{\morse}}}$ 
  & SCS-layer\\
  \hline
  $\strtl{\strcns{x}{\astr}} \red \astr$ & \\[-0.0ex]
  $\strzip{\strcns{x}{\astr}}{\bstr} \red \strcns{x}{\strzip{\bstr}{\astr}}$ 
  & SFS-layer \\[-0.0ex]
  $\strinv{\strcns{x}{\astr}} \red \strcns{\funap{\datsymb{i}}{x}}{\strinv{\astr}}$ & \\
  \hline
  $\funap{\datsymb{i}}{\datsymb{0}} \red \datsymb{1} \quad\quad \funap{\datsymb{i}}{\datsymb{1}} \red \datsymb{0}$
  & data-layer \\
  \hline
\end{tabular}
               }
\end{center}
  %
  \mbox{}\\[0.5ex]
  %
\end{example}
 
\begin{flushleft}
  \scalebox{1.1}{This is a \emph{productive} pSCS,
                 obtaining the \emph{Thue-Morse sequence}:}\\[1.5ex]
  \hspace*{\fill}
  \scalebox{1.1}{
  $
  \morse \infred \strcns{0}{\strcns{1}{\strcns{1}\strcns{0}{
                 \strcns{1}{\strcns{0}{\strcns{0}\strcns{1}{
                 \strcns{1}{\strcns{0}{\strcns{0}\strcns{1}{
                 \strcns{0}{\strcns{1}{\strcns{1}\strcns{0}{
                   \ldots}}}}}}}}}}}}
  $
                  }
\end{flushleft}

\end{frame}%%%%%%%%%%%%%%%%%%%%%%%%%%%%%%%%%%%%%%%%%%%%%%%%%%%%%%%%%%%%%%


\begin{frame}%%%%%%%%%%%%%%%%%%%%%%%%%%%%%%%%%%%%%%%%%%%%%%%%%%%%%%%%%%%%
  \frametitle{Pure Stream Constant Specifications}


\begin{example}
  %
  \vspace*{2.5ex}
  %
\begin{center}
\scalebox{1.1}{
\renewcommand{\arraystretch}{1.25}
\begin{tabular}{|c|l|}
  \hline
  $\joerg 
      = \strcns{\datsymb{0}}{\strcns{\datsymb{1}}{\strev{\joerg}}}$
  & SCS-layer\\
  \hline
  $\strev{\strcns{x}{\astr}} \red \strcns{x}{\strod{\astr}}$ & \\[-0.0ex]
  $\strod{\strcns{x}{\astr}} \red \strev{\astr} $ 
  & SFS-layer\\[-0.0ex]
  \hline
  \phantom{
    $\funap{\datsymb{i}}{\datsymb{0}} \red \datsymb{1}\quad\quad \funap{\datsymb{i}}{\datsymb{1}} \red \datsymb{0}$
          }
  & data-layer \\
  \hline
\end{tabular}
               }
\end{center}
  %
  \mbox{}\\[0.5ex]
  %
\end{example}

\vspace*{1ex}
%
\begin{flushleft}
  \scalebox{1.25}{This pSCS is \emph{not productive}: \mbox{}
  $
  \joerg \infred \strcns{0}{\strcns{1}{\strcns{0}\strcns{0}{
                                           \sstrev^{\omega}}}}
  $
                  }
\end{flushleft}
 
\end{frame}%%%%%%%%%%%%%%%%%%%%%%%%%%%%%%%%%%%%%%%%%%%%%%%%%%%%%%%%%%%%%%


\begin{frame}%%%%%%%%%%%%%%%%%%%%%%%%%%%%%%%%%%%%%%%%%%%%%%%%%%%%%%%%%%%
  \frametitle{$\joerg %\red \strcns{0}{\strcns{1}{\strev{\strcJ}}}
                      \infred 
                       \strcns{0}{\strcns{1}{\strcns{0}\strcns{0}{
                                  \sstrev^{\omega}}}}$}
  \vspace*{-3.75ex}
  %
  \begin{align*}
    %
    %\uncover<1->{
    \strcJ  
      & \red 
        \strcns{0}{\strcns{1}{\strev{\strcJ}}}  
    %             }
        %
                                                                  \\[0.2ex]
        %
    %\uncover<2->{
    \strev{\strcJ} 
      & \red
        \strev{\strcns{0}{\strcns{1}{\strev{\strcJ}}}}
                                                                    \\[-0.35ex]
      & \red
        \strcns{0}{\strod{\strcns{1}{\strev{\strcJ}}}}
                                                                    \\[-0.35ex]
      & \red
        \strcns{0}{\strev{\strev{\strcJ}}} 
    %             }
        %
                                                                  \\[0.2ex]
        %
    %\uncover<3->{
    \strevsq{\strcJ} \equiv
    \strev{\strev{\strcJ}}
      & \mred 
        \strev{ \strcns{0}{\strev{\strev{\strcJ}}} }
                                                                    \\[-0.35ex]
      & \red
        \strcns{0}{\strod{\strevsq{\strcJ}}} 
    %             }
        %
                                                                  \\[0.2ex]
        %
    %\uncover<4->{
    \strod{\strevsq{\strcJ}} 
      & \mred 
        \strod{\strcns{0}{\strod{\strevsq{\strcJ}}}}
                                                                    \\[-0.35ex]
      & \red
        \strev{\strod{\strevsq{\strcJ}}} 
    %             }
        %
                                                                  \\[0.2ex]
        %
    %\uncover<5->{
    \strod{\strevsq{\strcJ}}
      & \mred
        \strev{\strod{\strevsq{\strcJ}}}
                                                                    \\[-0.35ex]
      & \mred
        \strevsq{\strod{\strevsq{\strcJ}}}
                                                                    \\[-0.35ex]
      & \mred \ldots
        \mred \strevn{\strod{\strevsq{\strcJ}}} 
        \mred \ldots
                                                                    \\[-0.35ex]
      & \infred 
        \sstrevom 
    %             }
        %
  \end{align*}

  \vspace*{-2ex}
  %\uncover<6->{
  Hence: $\strcJ \infred 
                     \strcns{0}{\strcns{1}{\strcns{0}\strcns{0}{
                                           \sstrev^{\omega}}}}$.
  %            } 

\end{frame}%%%%%%%%%%%%%%%%%%%%%%%%%%%%%%%%%%%%%%%%%%%%%%%%%%%%%%%%%%%%%%
     

\begin{frame}%%%%%%%%%%%%%%%%%%%%%%%%%%%%%%%%%%%%%%%%%%%%%%%%%%%%%%%%%%%%
  \frametitle{Pure Stream Constant Specifications}

\begin{example}
  %
  \vspace*{1.75ex}
  %
\begin{center}
\vspace*{-0.5ex}
\scalebox{1}{
\renewcommand{\arraystretch}{1.25}
\begin{tabular}{|c|l|}
  \hline
  $ \dimitri' \red \strcns{0}{\strcns{1}{\strcns{1}{
                  \strzip{\stradd{\strtl{\dimitri'}}{\strtl{\strtl{\dimitri'}}}}
                         {\strcon{E}} }}} $
  & SCS-layer\\
  $ \strcon{E} \red \strev{\strtl{\dimitri'}} $ & \\
  \hline
  $\strtl{\strcns{x}{\astr}} \red \astr$ & \\[-0.0ex]
  $\strev{\strcns{x}{\astr}} \red \strcns{x}{\strod{\astr}}$ & \\[-0.0ex]
  $\strod{\strcns{x}{\astr}} \red \strev{\astr} $ 
  & SFS-layer\\[-0.0ex]
  $ \stradd{ \strcns{x}{\astr} }{ \strcns{y}{\bstr} }
         \red   \strcns{\strad{x}{y}}
                           { \stradd{\astr}{\bstr} } $ & \\[-0.0ex]
  $\strzip{\strcns{x}{\astr}}{\bstr} \red \strcns{x}{\strzip{\bstr}{\astr}}$
  & \\[-0.0ex]
  \hline
  $ \strad{x}{0} \red   x  $ \qquad\qquad
  $ \strad{x}{\strsucc{y}}   \red   \strsucc{\strad{x}{y}} $
  & data-layer \\
  \hline
\end{tabular}
               }
\end{center}
  %
  %\mbox{}\\[-1.5ex]
  %
\end{example}
 
\begin{flushleft}
  \scalebox{1}{This pSCS is \emph{productive}, we obtain:}\\[1ex]
  \hspace*{\fill}
  \scalebox{1}{
  $
  \dimitri' \infred 
      \strcns{0}{\strcns{1}{\strcns{1}\strcns{2}{\strcns{1}{
                 \strcns{3}{\strcns{2}\strcns{3}{\strcns{3}{
                 \strcns{4}{\strcns{3}\strcns{5}{\strcns{4}{
                 \strcns{5}{\strcns{5}\strcns{6}{\strcns{5}{
                 \strcns{7}{\strcns{6}\strcns{7}{\strcns{7}{\ldots}
                 }}}}}}}}}}}}}}}
  $
                  }
\end{flushleft}

\end{frame}%%%%%%%%%%%%%%%%%%%%%%%%%%%%%%%%%%%%%%%%%%%%%%%%%%%%%%%%%%%%%%


\begin{frame}%%%%%%%%%%%%%%%%%%%%%%%%%%%%%%%%%%%%%%%%%%%%%%%%%%%%%%%%%%%%
  \frametitle{Pure Stream Constant Specifications}

\begin{definition}
  %
  A TRS 
  $\atrs = \pair{\Sd\uplus\Ssf\uplus\Ssc\uplus\{\sstrcns\}}{\Rd\uplus \Rsf\uplus\Rsc}$ 
  is a \magenta{pure recursive stream specification (SCS)} 
  %% (for each of the symbols in $\asig_{\mit{sc}}$)
  if:
  %
  \begin{enumerate}
    %
    \item  $\trssub{T}{0} \defdby \pair{\Sd\uplus\Ssf\uplus\{\sstrcns\}}{\Rd\uplus\Rsf}$  
      is a weakly guarded SFS.%\pause
      %% (for each of the symbols in $\Ssf$).
      %
    \item $ \Ssc = \{ \rootsc,\ldots,\strcon{M_n} \} $
      set of \emph{stream constant symbols},\\
      where $\rootsc$ is called \emph{the root of $\atrs$};\\
      %
      $ \Rsc = \{ \rho_{\strcon{M_i}} \where i\in\{0,1,\ldots,n\} \} $,
      where $\rho_{\strcon{M_i}}$ 
      the \emph{defining rule for $\strcon{M_i}$}:
      %
      \begin{align*}
        %
        \strcon{M_i} &\red \cxtap{C_i}{\strcon{M_0},\ldots,\chocolate{\strcon{M_n}}} 
        &
        \text{($C_i$ an $n$-ary stream context in $\trssub{T}{0}$).}
        %
      \end{align*}
      %
      %where $C_i$ an $n$-ary stream context in $\trssub{T}{0}$.
      %
  \end{enumerate}
  %
  \vspace*{-2ex}
  We call $\Rd$, $\Rsf$, and $\Rsc$ the
  \emph{data-}, \emph{\SFS-}, and \emph{\SCS-layer of $\atrs$}, resp..%
  %\pause{}
  %
\end{definition}

  
  %% \vspace*{1ex}
  %% Note: A pure \SCS\ is `orthogonal' 
  %% w.r.t.\ the rules in the SFS- and SCS-layers
  %% (no redex-pattern overlaps).
  %
\end{frame}%%%%%%%%%%%%%%%%%%%%%%%%%%%%%%%%%%%%%%%%%%%%%%%%%%%%%%%%%%%%%%


\begin{frame}%%%%%%%%%%%%%%%%%%%%%%%%%%%%%%%%%%%%%%%%%%%%%%%%%%%%%%%%%%%%
  \frametitle{Production of a Term. Productivity of an SCS.}


  Let $\conat \defdby \nat\cup\{\infty\}$ the
  \emph{coinductive natural numbers}. 


\begin{definition}
  %
  Let $\atrs = \pair{\asig}{R}$ a pure \SCS. 
  %
  \begin{itemize}
    %
    \item The \magenta{production} $\chocolate{\terprd{\atrs}{\darkcyan{t}}}$
      of a stream term $\darkcyan{t}  \in \ter{\asig}$: % is 
       %% the supremum of the number of data elements $\darkcyan{t}$ 
       %% can `produce':
      %
      \begin{equation*}
        %
        \chocolate{\terprd{\atrs}{\darkcyan{t} }}
          \defdby \supremum{\{\chocolate{n} \in \nat \mid  
          \darkcyan{t} \mred \lstcns{\adattrm_1}{\lstcns{\ldots}{\lstcns{\adattrm_{\chocolate{n}}}{t'}}}\}} 
          \in\conat \; .
           %
      \end{equation*}
      %
      \vspace*{-1ex}
      %
    \item $\atrs$ is called \magenta{productive} if
      $\terprd{\atrs}{\rootsc} = \infty$.
      %
  \end{itemize}
  %
\end{definition}\pause{}


\vspace*{1ex}
%
\begin{proposition}
  %
  A pure \SCS~$\atrs$ is productive if and only if\/
  $ \rootsc \infred 
           \lstcns{\adattrm_1}{\lstcns{\adattrm_2}
                              {\lstcns{\adattrm_3}
                              {\lstcns{\adattrm_4}{\ldots}}}} $.
  %
\end{proposition}

\end{frame}%%%%%%%%%%%%%%%%%%%%%%%%%%%%%%%%%%%%%%%%%%%%%%%%%%%%%%%%%%%%%%


%------------------------------------------------------------------------
%------------------------------------------------------------------------
\section{Modelling with Nets}
%------------------------------------------------------------------------
%------------------------------------------------------------------------


%------------------------------------------------------------------------
\subsection{Pebbleflow Nets}
%------------------------------------------------------------------------

\begin{frame}%%%%%%%%%%%%%%%%%%%%%%%%%%%%%%%%%%%%%%%%%%%%%%%%%%%%%%%%%%%%
  \frametitle{Modelling SCSs with Pebbleflow Nets}

  \begin{itemize}%[<+->]
    \item 
      Kahn (1974): Networks are devices 
      for computing least fixed points of systems of equations.
  \end{itemize}

\chocolate{Pebbleflow Nets}:
  \begin{itemize}%[<+->]
    \item
      Stream elements %(+ constructors) 
      are abstracted 
      from in favour of `pebbles':
      $\trnsl{\strcns{\adattrm}{\astrtrm}} = \netpeb{\trnsl{\astrtrm}}$.
      %
    \item
      An SCS is modelled by a pebbleflow net:\\
      Evaluation of an SCS is modelled by
      the flow of pebbles in a net.
      %
    \item
     A stream definition is productive if and only if 
     the net associated to it
     generates an infinite chain of pebbles.
    \item 
      Elements are: \emph{fans}, \emph{meets}, \emph{boxes} and \emph{gates},
        \emph{sources}, \emph{wires}. 
      %
  \end{itemize}

\end{frame}%%%%%%%%%%%%%%%%%%%%%%%%%%%%%%%%%%%%%%%%%%%%%%%%%%%%%%%%%%%%%%


\begin{frame}%%%%%%%%%%%%%%%%%%%%%%%%%%%%%%%%%%%%%%%%%%%%%%%%%%%%%%%%%%%%
\frametitle{Fan}

  \begin{center}
  \fbox{\scalebox{.8}{\input{\figpath/fan.pstex_t}}}
  \\
  \mbox{}\\
  \begin{itemize}  
    \item %A \emph{fan} 
    \emph{duplicates} an incoming pebble along its output ports
    \item
    explicit \emph{sharing} device
    \item
    enables construction of \emph{cyclic nets}
    \item
    used to implement \emph{recursion}, in particular \emph{feedback}
  \end{itemize}
  \end{center}

\end{frame}%%%%%%%%%%%%%%%%%%%%%%%%%%%%%%%%%%%%%%%%%%%%%%%%%%%%%%%%%%%%%%


\begin{frame}%%%%%%%%%%%%%%%%%%%%%%%%%%%%%%%%%%%%%%%%%%%%%%%%%%%%%%%%%%%%
\frametitle{Recursion/Feedback}

  \begin{center}
  \fbox{\scalebox{.7}{\input{\figpath/recfan.pstex_t}}}
  \\
  \mbox{}\\
  $\netrec{\anam}{\netpeb{\anet(\avar)}}
   \red \netpeb{\netrec{\anam}{\anet(\netpeb{\avar})}}$
  \end{center}

\end{frame}%%%%%%%%%%%%%%%%%%%%%%%%%%%%%%%%%%%%%%%%%%%%%%%%%%%%%%%%%%%%%%


\begin{frame}%%%%%%%%%%%%%%%%%%%%%%%%%%%%%%%%%%%%%%%%%%%%%%%%%%%%%%%%%%%%
\frametitle{Meet}

  \begin{center}
  %
  \fbox{\scalebox{.8}{\input{\figpath/meet.pstex_t}}}
  \\
  \mbox{}\\
  \mbox{}\\
  $\trmmeet{\trmpeb{\iatrm{1}}}{\trmpeb{\iatrm{2}}} 
   \red \trmpeb{\trmmeet{\iatrm{1}}{\iatrm{2}}}$
   %
  \end{center}

\end{frame}%%%%%%%%%%%%%%%%%%%%%%%%%%%%%%%%%%%%%%%%%%%%%%%%%%%%%%%%%%%%%%


\begin{frame}%%%%%%%%%%%%%%%%%%%%%%%%%%%%%%%%%%%%%%%%%%%%%%%%%%%%%%%%%%%%
\frametitle{Box}

  \begin{center}
    %
    \newcommand{\ah}{\iosqcns{\ioout}{\aiosq}}
    \fbox{\scalebox{.85}{\input{\figpath/box1.pstex_t}}} 
    %
    \hspace*{6em}
    %
    \renewcommand{\ah}{\iosqcns{\ioin}{\aiosq}}
    \fbox{\scalebox{.85}{\input{\figpath/box2.pstex_t}}} 
    \\
    \mbox{}\\
    \mbox{}\\
    %
    $\trmbox{\iosqcns{\ioout}{\aiosq}}{\atrm} \red \trmpeb{\trmbox{\aiosq}{\atrm}}$
    %
    \hspace*{4.75em}
    %
    $\trmbox{\iosqcns{\ioin}{\aiosq}}{\trmpeb{\atrm}} \red \trmbox{\aiosq}{\atrm}$
    %
  \end{center}


  \begin{itemize}
    \item
      $\ioout$ : a ready state for an output pebble
    \item
      $\ioin$ : a requirement for an input pebble
    \item
      $\aiosq$ : an infinite sequence over $\io$
  \end{itemize}


\end{frame}%%%%%%%%%%%%%%%%%%%%%%%%%%%%%%%%%%%%%%%%%%%%%%%%%%%%%%%%%%%%%%



\begin{frame}%%%%%%%%%%%%%%%%%%%%%%%%%%%%%%%%%%%%%%%%%%%%%%%%%%%%%%%%%%%%
  \frametitle{\ioseqs}

\begin{itemize}
  \item
    %boxes contain \emph{\ioseqs}:
    %infinite sequences over $\io$
    %that 
    \emph{\ioseqs} contain \emph{infinitely many $\ioout$'s}:
  %\item
    \[
      \iostr 
      \defdby 
      \{ 
         \aiosq\in\str{\io} 
         \where 
         \myall{n}{\myex{m\ge n}{\,\strnth{\aiosq}{m} = \ioout}} 
      \}
    \]
  \vspace*{-2.5ex}
  \item
    $\aiosq\in\iosq$ is \emph{rational}
    if there exist $\alst,\clst \in \iolst$ such that 
    $\aiosq = \alst \lstcyc{\clst} = \alst\clst\clst\clst\ldots$
    ($\clst$ non-empty) 
  \vspace{.75ex}
  \item
    \ioseqs\ model \emph{quantitative behaviour} of
    stream functions
  \end{itemize}

%% 
%% \begin{definition}\label{def:iosq}
%%   The set \chocolate{$\iostr$} of \magenta{\ioseqs}
%%   is the set of 
%%   infinite sequences over the alphabet~$\io$
%%   that contain an infinite number of~$\ioout$'s:
%%   %
%%   \begin{align*}
%%     \chocolate{\iostr} 
%%     &\defdby 
%%     \{ 
%%       \aiosq\in\str{\io} 
%%       \where \forall{n}{\exists{m}{ \;
%%                          \funap{\aiosq}{n+m} = \ioout}} 
%%     \}
%%   \end{align*}
%% 
%%   An \ioseq~$\aiosq\in\iosq$ is called \magenta{rational}
%%   if there exist lists $\alst,\clst \in \iolst$ such that
%%   $\aiosq = \alst \lstcyc{\clst}$, 
%%   %% %
%%   %% \begin{itemize}
%%   %%   %
%%   %%   \item $\clst$ is not the empty list, and
%%   %%     %
%%   %%   \item $\lstcyc{\clst}$ denotes the infinite sequence
%%   %%     $\clst\clst\clst\ldots$.
%%   %%     %
%%   %% \end{itemize}
%%   %
%%   where $\clst$ is not empty.\\
%%   % and denotes the infinite sequence
%%   % $\clst\clst\clst\ldots$.
%%   The pair $\pair{\alst}{\clst}$
%%   is called a \emph{rational representation} of $\aiosq$.\\
%%   %%
%%   %% $\chocolate{\iosqrat}\subseteq\chocolate{\iosq}$
%%   %% we denote the set of \magenta{rational \ioseqs}.
%%   %
%%   And we define: 
%%   %
%%   \begin{align*}
%%     %
%%     \chocolate{\iosqrat}
%%       \defdby
%%      \{ 
%%        \aiosq\in\chocolate{\iosq}
%%      \where
%%        \text{$\aiosq$ is rational}
%%      \} \; . 
%%      %
%%   \end{align*}
%%   %
%% \end{definition}

\end{frame}%%%%%%%%%%%%%%%%%%%%%%%%%%%%%%%%%%%%%%%%%%%%%%%%%%%%%%%%%%%%%%



\begin{frame}%%%%%%%%%%%%%%%%%%%%%%%%%%%%%%%%%%%%%%%%%%%%%%%%%%%%%%%%%%%%
\frametitle{Gates}

  \begin{center}
  %
  \fbox{\scalebox{0.8}{\input{\figpath/gate.pstex_t}}}
  \\
  \mbox{}\\
  \mbox{}\\
  A gate for modelling $\arfs$-ary stream functions.
  \\
  \mbox{}\\
  \mbox{}\\
  $\funap{\strmmeet}{\trmbox{\aiosq_1}{\cxthol_1}, 
                     \ldots, \trmbox{\aiosq_{\arfs}}{\cxthol_{\arfs}}}$
  %
  \end{center}

\end{frame}%%%%%%%%%%%%%%%%%%%%%%%%%%%%%%%%%%%%%%%%%%%%%%%%%%%%%%%%%%%%%%



\begin{frame}%%%%%%%%%%%%%%%%%%%%%%%%%%%%%%%%%%%%%%%%%%%%%%%%%%%%%%%%%%%%
  \frametitle{Source}

  \begin{center}
  \newcommand{\ah}{\conatsuc{k}}
  \fbox{\scalebox{.8}{\input{\figpath/source.pstex_t}}}
  \\
  \mbox{}\\
  $\netsrc{\conatsuc{k}} \red \netpeb{\netsrc{k}}$
  \end{center}

\end{frame}%%%%%%%%%%%%%%%%%%%%%%%%%%%%%%%%%%%%%%%%%%%%%%%%%%%%%%%%%%%%%



\begin{frame}%%%%%%%%%%%%%%%%%%%%%%%%%%%%%%%%%%%%%%%%%%%%%%%%%%%%%%%%%%%%
  \frametitle{Term Representations of Nets}

  \begin{definition}
    %
    Let $\nam$ be a set of variables.\\[0.5ex]
  
    The set $\trm$ of \magenta{terms for pebbleflow nets}
    is generated by: 
    \[
      %\trm \ni 
      \atrm
      \BNFis
      \trmsrc{k}
      \BNFor
      \avar
      \BNFor
      \trmpeb{\atrm}
      \BNFor
      \trmbox{\aiosq}{\atrm}
      \BNFor
      \trmrec{\anam}{\atrm}
      \BNFor
      \trmmeet{\atrm}{\atrm}
    \]
    where $\anam\in\nam$, $\aiosq\in\iosq$,
    and $k \in \conat \defdby \nat \cup \{\conattop\}$.
    %
  \end{definition}

\end{frame}%%%%%%%%%%%%%%%%%%%%%%%%%%%%%%%%%%%%%%%%%%%%%%%%%%%%%%%%%%%%%%



%------------------------------------------------------------------------
\subsection{A Rewrite System for Pebbleflow. Ariya's Tool.}
%------------------------------------------------------------------------


\begin{frame}%%%%%%%%%%%%%%%%%%%%%%%%%%%%%%%%%%%%%%%%%%%%%%%%%%%%%%%%%%%%
  \frametitle{Pebbleflow}

\setcounter{equation}{0}
\renewcommand{\theequation}{$\pebrul${\arabic{equation}}}
%
\begin{definition}
  %
  The \emph{pebbleflow rewrite relation \chocolate{$\spebred$}}
  is defined as:
  %
  \begin{align*}
    %
    \trmmeet{\trmpeb{\atrm_1}}{\trmpeb{\atrm_2}}
    &\red \trmpeb{\trmmeet{\atrm_1}{\atrm_2}} 
      %% \label{def:pebred:rule:meet}
    \\
    \trmrec{\anam}{\trmpeb{\atrm(\avar)}}
    &\red 
    \trmpeb{\trmrec{\anam}{\atrm(\trmpeb{\trmvar{\anam}})}}
      %% \label{def:pebred:rule:feedback}
    \\
    \trmbox{(\iosqcns{\ioout}{\aiosq})}{\atrm}
    &\red \trmpeb{\trmbox{\aiosq}{\atrm}} 
      %% \label{def:pebred:rule:boxP}
    \\
    \trmbox{(\iosqcns{\ioin}{\aiosq})}{\trmpeb{\atrm}}
    &\red \trmbox{\aiosq}{\atrm}
      %% \label{def:pebred:rule:boxM}
    \\
    \trmsrc{\conatsuc{{k}}}
    &\red
    \trmpeb{\trmsrc{{k}}}
      %% \label{def:pebred:rule:source}
  \end{align*}
  %
\end{definition}


%\chocolate{$\spebred$} is an orthogonal CRS, and hence:
\vspace*{0.75ex}
%
\begin{theorem}
  %
  The rewrite relation \chocolate{$\spebred$} is confluent.
  %
\end{theorem}

\end{frame}%%%%%%%%%%%%%%%%%%%%%%%%%%%%%%%%%%%%%%%%%%%%%%%%%%%%%%%%%%%%%%



\begin{frame}%%%%%%%%%%%%%%%%%%%%%%%%%%%%%%%%%%%%%%%%%%%%%%%%%%%%%%%%%%%%
  \frametitle{Pebbleflow Tool: Watching Streams Grow}

  \vspace*{0.5ex}
  %
  \begin{itemize}
    %
    \item  \emph{Net visualization tool} (Java applet) 
      by \href{http://www.few.vu.nl/~ariya/}{Ariya Isihara}\\[0.75ex]
      %
      click\hspace*{1pt}\&\hspace*{1pt}play:
      \href{http://infinity.few.vu.nl/productivity}%
           {\emph{\tt http://infinity.few.vu.nl/productivity}} 
      %
  \end{itemize}


    %\vspace*{0.5ex}
    %
    \begin{align*}
      %
      \morse 
      &= \strcns{\datsymb{0}}{\strzip{\strinv{\morse}}{\strtl{\morse}}}
      \\
      \href{http://infinity.few.vu.nl/productivity/pf/index.php?load=nets/morse.net.gz}{\trnsl{\morse}}
           %\urlprefix/pf/morse_alternative.html
      &= 
      \netrec{\anam}{\netpeb{\netmeet
        {\netbox{\ios{}{-++}}{\netbox{\ios{}{-+}}{\netvar{\anam}}}}
        {\netbox{\ios{}{+-+}}{\netbox{\ios{-}{-+}}{\netvar{\anam}}}}
      }}
      \\[2.5ex]
      %
      \joerg 
      &= \strcns{\datsymb{0}}{\strcns{\datsymb{1}}{\strev{\joerg}}}
      \\
      \href{http://infinity.few.vu.nl/productivity/pf/index.php?load=nets/morse.net.gz}{\trnsl{\joerg}}
        &= 
        \netrec{\anam}{\netpeb{\netpeb{
          \netbox{\ios{}{-+-}}{\netvar{\anam}}}
        }}
      \\[2.5ex]
      %
      \dimitri 
        &= \strcns{\datsymb{0}}{\strcns{\datsymb{1}}{\strcns{\datsymb{0}}{
                      \strzip{\stradd{\strtl{\dimitri}}{\strtl{\strtl{\dimitri}}}}
                             {\strev{\strtl{\dimitri}}} }}}
      \\
      \href{http://infinity.few.vu.nl/productivity/pf/index.php?load=nets/D.net.gz}{\trnsl{\dimitri}}
        &= 
        \trmrec{D}{
          \trmpeb{\trmpeb{\trmpeb{
            \bfunap{\trnsl{\sstrzip}}{
              \bfunap{\trnsl{\sstradd}}{\funap{\trnsl{\sstrtl}}{D}}{\funap{\trnsl{\sstrtl}}{\funap{\trnsl{\sstrtl}}{D}}}}{
              \funap{\trnsl{\sstrev}}{\funap{\trnsl{\sstrtl}}{D}}
            }
          }}}}
      %
    \end{align*}
    
\end{frame}%%%%%%%%%%%%%%%%%%%%%%%%%%%%%%%%%%%%%%%%%%%%%%%%%%%%%%%%%%%%



\begin{frame}%%%%%%%%%%%%%%%%%%%%%%%%%%%%%%%%%%%%%%%%%%%%%%%%%%%%%%%%%%%%%%
  \frametitle{Production of a Net}

\renewcommand{\strmbhv}{\chocolate{\pi}}
%
\begin{definition}
  %
  The \emph{production} $\chocolate{\netprd{\darkcyan{\atrm}}}$ 
  of a net $\darkcyan{\atrm}\in\trm$:
   % is the supremum of the number of pebbles the net can `produce':
  \[
    \chocolate{\netbhv{\darkcyan{\atrm}}} 
      \defdby 
    \supremum{\{ n\in\nat \mid 
                 \darkcyan{\atrm} 
                 \mathrel{\chocolate{\mpebred}} \trmpebn{n}{\atrm'}\}}\:\text{.}
  \]
  %
\end{definition}
%
\renewcommand{\strmbhv}{{\pi}}

\end{frame}%%%%%%%%%%%%%%%%%%%%%%%%%%%%%%%%%%%%%%%%%%%%%%%%%%%%%%%%%%%%%%




%------------------------------------------------------------------------
\subsection{Translating Pure Stream Specifications}
%------------------------------------------------------------------------


\begin{frame}%%%%%%%%%%%%%%%%%%%%%%%%%%%%%%%%%%%%%%%%%%%%%%%%%%%%%%%%%%%%
  \frametitle{Translation of Unary Stream Functions into I/O-Seq's}


\begin{example}

  \begin{align*}
    \strtl{\strcns{x}{\astr}} &= \astr \\
    \strev{\strcns{x}{\astr}} &= \strcns{x}{\strod{\astr}} \\
    \strod{\strcns{x}{\astr}} &= \strev{\astr} \\
      \strdup{\strcns{x}{\astr}} 
      &= \strcns{x}{\strcns{x}{\strdup{\astr}}}
  \end{align*}

  \vspace*{-4.5ex}
  %
  \begin{align*}
    \trnsli{\sstrtl}{1}
    & = \iosqcns{\ioin}{\iaiosq{\msf{id}}}
    &
    \trnsli{\sstrtl}{1}
    & = \iosqcns{\ioin}{\iosqcyc{\iosqcns{\ioin}{\iosqcns{\ioout}{\nix}}}}
    \\
    \trnsli{\sstrev}{1}
    & = \iosqcns{\ioin}{\iosqcns{\ioout}{\trnsli{\sstrod}{1}}}
    &
    \trnsli{\sstrev}{1}
    & = \iosqcyc{\iosqcns{\ioin}{\iosqcns{\ioout}{\iosqcns{\ioin}{\nix}}}}
    \\
    \trnsli{\sstrod}{1}
    & = \iosqcns{\ioin}{\trnsli{\sstrev}{1}}
    &
    \trnsli{\sstrod}{1}
    & = \iosqcyc{\iosqcns{\ioin}{\iosqcns{\ioin}{\iosqcns{\ioout}{\nix}}}}
    \\ 
    \trnsli{\sstrdup}{1}
    & = \iosqcns{\ioin}{\iosqcns{\ioout}{\iosqcns{\ioout}\trnsli{\sstrdup}{1}}}
    &
    \trnsli{\sstrdup}{1}
    & = \iosqcyc{\iosqcns{\ioin}{\iosqcns{\ioout}{\iosqcns{\ioout}{\nix}}}}
  \end{align*}

\end{example}

\end{frame}%%%%%%%%%%%%%%%%%%%%%%%%%%%%%%%%%%%%%%%%%%%%%%%%%%%%%%%%%%%%%%



\begin{frame}%%%%%%%%%%%%%%%%%%%%%%%%%%%%%%%%%%%%%%%%%%%%%%%%%%%%%%%%%%%%
  \frametitle{Translation of Stream Functions into Gates}

 \begin{example}
  %The definition of $\sstrzip$ :
  \[
    \strzip{\strcns{x}{\astr}}{\bstr}
    = \strcns{x}{\strzip{\bstr}{\astr}}
  \]
  %giving rise to the dependency cycle:
  %\[
  %  \strzip{\strcns{x}{\astr}}{\strcns{y}{\bstr}}
  %  \red
  %  \strcns{x}{\strzip{\strcns{y}{\bstr}}{\astr}}
  %  \red
  %  \strcns{x}{\strcns{y}{\strzip{\astr}{\bstr}}}
  %\]
  %is translated as follows:
  %\[
  %  \bfunap{\trnsl{\sstrzip}}{\ianet{1}}{\ianet{2}}
  %  = 
  %  \netmeet
  %    {\netbox{\trnsli{\sstrzip}{1}}{\ianet{1}}}
  %    {\netbox{\trnsli{\sstrzip}{2}}{\ianet{1}}}
  %\]
  %where:
  %\vspace{-2ex}
  \[
   \trnsl{\sstrzip}
   = \netmeetn{2}{
       \netbox{\trnsli{\sstrzip}{1}}{\cxthol_1},
       \netbox{\trnsli{\sstrzip}{2}}{\cxthol_2}
     }\:,
  \]
  where:
  \begin{align*}
    \trnsli{\sstrzip}{1}
    &= \iosqcns{\ioin}{\iosqcns{\ioout}{\trnsli{\sstrzip}{2}}}
    &
    \trnsli{\sstrzip}{1}
    &= \iosqcyc{\iosqcns{\ioin}{\iosqcns{\ioout}{\iosqcns{\ioout}{\nix}}}}
    \\
    \trnsli{\sstrzip}{2}
    &= \iosqcns{\ioout}{\trnsli{\sstrzip}{1}}
    &
    \trnsli{\sstrzip}{2}
    &= \iosqcyc{\iosqcns{\ioout}{\iosqcns{\ioin}{\iosqcns{\ioout}{\nix}}}}
  \end{align*}





\end{example}
  
\end{frame}%%%%%%%%%%%%%%%%%%%%%%%%%%%%%%%%%%%%%%%%%%%%%%%%%%%%%%%%%%%%%%



\begin{frame}%%%%%%%%%%%%%%%%%%%%%%%%%%%%%%%%%%%%%%%%%%%%%%%%%%%%%%%%%%%%
  \frametitle{Translation of Stream Constants into Nets}

\vspace*{-0.5ex}
%
\begin{example}
  %
  \newcommand{\fusctl}{$\trnsli{\sstrtl}{1}$}
  \newcommand{\fusczp}{$\trnsl{\sstrzip}$}
  \newcommand{\fusczpa}{$\trnsli{\sstrzip}{1}$}
  \newcommand{\fusczpb}{$\trnsli{\sstrzip}{2}$}
  \newcommand{\fuscad}{$\trnsl{\sstradd}$}
  \newcommand{\fuscada}{$\trnsli{\sstradd}{1}$}
  \newcommand{\fuscadb}{$\trnsli{\sstradd}{2}$}
  \newcommand{\fuscev}{$\trnsli{\sstrev}{1}$}
  %
  \begin{center}
    %
    \vspace*{-2.5ex}
    \[
    \dimitri \red \strcns{0}{\strcns{1}{\strcns{0}{
                    \strzip{\stradd{\strtl{\dimitri}}{\strtl{\strtl{\dimitri}}}}
                           {\strev{\strtl{\dimitri}}} }}} 
    \] 
  %
  \fbox{\scalebox{.5}{\input{\figpath/fusc.pstex_t}}}
  %
  \[
  \href{http://infinity.few.vu.nl/productivity/pf/index.php?load=nets/D.net.gz}%
    {
    \trnsl{\dimitri}
    }
    =
    \trmrec{D}{
      \trmpeb{\trmpeb{\trmpeb{
        \bfunap{\trnsl{\sstrzip}}{
          \bfunap{\trnsl{\sstradd}}{\funap{\trnsl{\sstrtl}}{D}}{\funap{\trnsl{\sstrtl}}{\funap{\trnsl{\sstrtl}}{D}}}}{
          \funap{\trnsl{\sstrev}}{\funap{\trnsl{\sstrtl}}{D}}
        }
      }}}
               }\vspace*{-2ex}
\]
  \end{center}
  %
\end{example}

\end{frame}%%%%%%%%%%%%%%%%%%%%%%%%%%%%%%%%%%%%%%%%%%%%%%%%%%%%%%%%%%%%%%


\begin{frame}%%%%%%%%%%%%%%%%%%%%%%%%%%%%%%%%%%%%%%%%%%%%%%%%%%%%%%%%%%%%
  \frametitle{Translation of Stream Constants into Nets}

\begin{definition}\label{def:trnsl:nets}
  %
  Let $\atrs = \pair{\Sd \uplus \Ssf \uplus \Ssc \uplus \{\sstrcns\}}{\Rd \uplus \Rsf \uplus \Rsc}$
  be a pure \SCS.\\
  For each $\astrcon\in\Ssc$ with rule 
  $\rho_{\msfm} \equiv {\astrcon \red \mit{rhs}_{\msfm}}$
  the translation $\trnsl{\astrcon} \defdby \trnsli{\astrcon}{\setemp}$ of $\astrcon$ 
  into a rational pebbleflow net is recursively def.\ by:
  %
  \begin{align*}
    \trnsli{\astrcon}{\alst} 
    &=
      \begin{cases}
      \trmrec{M}{\trnsli{\mit{rhs}_{\msfm}}{\setunion{\alst}{\{\astrcon\}}}} 
      &\text{if $\astrcon\not\in\alst$}
      \\
      M 
      &\text{if $\astrcon\in\alst$}
      \end{cases}
    \\
    \trnsli{\strcns{t}{u}}{\alst} 
    &= \trmpeb{\trnsli{u}{\alst}}
    \\
    \trnsli{
      \funap{\astrfun}{u_1,\ldots,u_{\arfs},t_1,\ldots,t_{\arfd}}}{
      \alst
    } 
    &=
    \funap{
      \trnsl{\astrfun}}{
      \trnsli{u_1}{\alst},\ldots,\trnsli{u_{\arfs}}{\alst}
    }
  \end{align*}
  %
  where $\alst$ denotes a set of stream constant symbols.
  %
\end{definition}

\end{frame}%%%%%%%%%%%%%%%%%%%%%%%%%%%%%%%%%%%%%%%%%%%%%%%%%%%%%%%%%%%%%%


%------------------------------------------------------------------------
\subsection{Preservation of Production}
%------------------------------------------------------------------------

\begin{frame}%%%%%%%%%%%%%%%%%%%%%%%%%%%%%%%%%%%%%%%%%%%%%%%%%%%%%%%%%%%%
\frametitle{Translation is Production Preserving}

\begin{theorem}
  %
  Let $ \atrs $ be a pure \SCS.
  Then it holds: $\netprd{\trnsl{\rootsc}} = \terprd{\atrs}{\rootsc}$.
  %
\end{theorem}\pause{}

\begin{proof}[General Idea]
  %
  \begin{itemize}
    %
    \item Going back and forth between rewrite sequences
      %
      \begin{center}
        $
        \trnsl{\rootsc} \mpebred \netpebn{n}{\anet}
        $
      \end{center}
      %
      on the pebbleflow net translation of $\rootsc$
      that produce $n$ pebbles, 
      and rewrite sequences 
      \begin{center}
        $
        \rootsc \mred_{\atrs} 
           \strcns{t_1}{\strcns{\ldots}{\strcns{t_n}{u}}}
        $
      \end{center}
      %
      in $\atrs$ that produce a prefix of $n$
      stream constructor.\pause{}
      %
    \item For this purpose establish a 
      \emph{bisimulation-like correspondence}
      between a $\mu$-term representations of the
      \SCS\ and the corresponding pebbleflow net
      using \emph{tracker symbols}.  
      %
  \end{itemize}
  \renewcommand{\qed}{}
  %
\end{proof}

\end{frame}%%%%%%%%%%%%%%%%%%%%%%%%%%%%%%%%%%%%%%%%%%%%%%%%%%%%%%%%%%%%%%



\begin{frame}%%%%%%%%%%%%%%%%%%%%%%%%%%%%%%%%%%%%%%%%%%%%%%%%%%%%%%%%%%%%
\frametitle{Preservation of Production (Idea)}

  
  \begin{flushleft}
    %
    \parbox[c]{100pt}{
      \scalebox{0.4}{
        \input{\figpathlocal/J_pfprod_1.pdf_t}
                      }
                       }
     %
     \hspace*{14ex}
     %
     \parbox[c]{100pt}{
       \scalebox{0.4}{
       \input{\figpathlocal/J_tprod_1.pdf_t}
                      }
                       }
      %
  \end{flushleft}

\end{frame}%%%%%%%%%%%%%%%%%%%%%%%%%%%%%%%%%%%%%%%%%%%%%%%%%%%%%%%%%%%%%%

\begin{frame}%%%%%%%%%%%%%%%%%%%%%%%%%%%%%%%%%%%%%%%%%%%%%%%%%%%%%%%%%%%%
\frametitle{Preservation of Production (Idea)}

  
  \begin{flushleft}
    %
    \parbox[c]{100pt}{
      \scalebox{0.4}{
        \input{\figpathlocal/J_pfprod_15.pdf_t}
                      }
                       }
     %
     \hspace*{14ex}
     %
     \parbox[c]{100pt}{
       \scalebox{0.35}{
       \input{\figpathlocal/J_tprod_15.pdf_t}
                      }
                       }
      %
  \end{flushleft}

\end{frame}%%%%%%%%%%%%%%%%%%%%%%%%%%%%%%%%%%%%%%%%%%%%%%%%%%%%%%%%%%%%%%

\begin{frame}%%%%%%%%%%%%%%%%%%%%%%%%%%%%%%%%%%%%%%%%%%%%%%%%%%%%%%%%%%%%
\frametitle{Preservation of Production (Idea)}

  
  \begin{flushleft}
    %
    \parbox[c]{100pt}{
      \scalebox{0.4}{
        \input{\figpathlocal/J_pfprod_2.pdf_t}
                      }
                       }
     %
     \hspace*{14ex}
     %
     \parbox[c]{100pt}{
       \scalebox{0.35}{
       \input{\figpathlocal/J_tprod_2.pdf_t}
                      }
                       }
      %
  \end{flushleft}

\end{frame}%%%%%%%%%%%%%%%%%%%%%%%%%%%%%%%%%%%%%%%%%%%%%%%%%%%%%%%%%%%%%%

\begin{frame}%%%%%%%%%%%%%%%%%%%%%%%%%%%%%%%%%%%%%%%%%%%%%%%%%%%%%%%%%%%%
\frametitle{Preservation of Production (Idea)}

  \begin{flushleft}
    %
    \parbox[c]{100pt}{
      \scalebox{0.4}{
        \input{\figpathlocal/J_pfprod_3.pdf_t}
                      }
                       }
     %
     \hspace*{12ex}
     %
     \parbox[c]{100pt}{
       \scalebox{0.325}{
       \input{\figpathlocal/J_tprod_3.pdf_t}
                      }
                       }
      %
  \end{flushleft}

\end{frame}%%%%%%%%%%%%%%%%%%%%%%%%%%%%%%%%%%%%%%%%%%%%%%%%%%%%%%%%%%%%%%

\begin{frame}%%%%%%%%%%%%%%%%%%%%%%%%%%%%%%%%%%%%%%%%%%%%%%%%%%%%%%%%%%%%
\frametitle{Preservation of Production (Idea)}
  
  \begin{flushleft}
    %
    \parbox[c]{100pt}{
      \scalebox{0.4}{
        \input{\figpathlocal/J_pfprod_4.pdf_t}
                      }
                       }
     %
     \hspace*{12ex}
     %
     \parbox[c]{100pt}{
       \scalebox{0.325}{
       \input{\figpathlocal/J_tprod_4.pdf_t}
                      }
                       }
      %
  \end{flushleft}

\end{frame}%%%%%%%%%%%%%%%%%%%%%%%%%%%%%%%%%%%%%%%%%%%%%%%%%%%%%%%%%%%%%%




%------------------------------------------------------------------------
%------------------------------------------------------------------------
\section{Deciding Productivity}
%------------------------------------------------------------------------
%------------------------------------------------------------------------


%------------------------------------------------------------------------
\subsection{Composition and Fixed Point}
%------------------------------------------------------------------------



\begin{frame}%%%%%%%%%%%%%%%%%%%%%%%%%%%%%%%%%%%%%%%%%%%%%%%%%%%%%%%%%%%%
  \frametitle{Production Function}

\renewcommand{\ssiosqbhv}{\chocolate{\beta}}
    %
    \newcommand{\bsgn}{$\iosqbhv{\aiosq}{n}$}
    \newcommand{\exBa}{\iosqcns{\ioout}{\iosqcns{\ioout}{\nix}}}
    \newcommand{\exBc}{\iosqcyc{\iosqcns{\ioin}{\iosqcns{\ioout}{\iosqcns{\ioin}{\iosqcns{\ioout}{\iosqcns{\ioin}{\nix}}}}}}}
    \newcommand{\exB}{\exBa \exBc}
    \newcommand{\aha}{\Big{\}}\darkcyan{n}}
    \newcommand{\ahb}{\chocolate{\beta}_{\sigma}(\darkcyan{n})}
    %
    %
    \begin{center}%{lr}
      %
      \fbox{\scalebox{0.8}{\input{\figpath/genbox.pstex_t}}}
      %
      \hspace{6em}
      %
      \fbox{\scalebox{0.8}{\input{\figpath/zz.pstex_t}}}
      %
    \end{center}
    %
    \vspace*{0.25ex}
    %
    \begin{flushleft}
      %
      \vspace*{-1ex}
      $\trmbox{\aiosq}{\trmpebn{\darkcyan{n}}{\atrm}} 
        \red \trmpebn{\chocolate{\beta}_{\aiosq}(\darkcyan{n})}{\trmbox{\aiosq'}{\atrm}}$
      %
      \hspace*{2.2em}
      %
      \parbox[t]{127.5pt}{\centering
        Graph of the\\[0.2ex] 
          \magenta{production function} $\siosqbhv{\aiosq}$\\[0.2ex]
        for $\aiosq = \exB$.}
      %
    \end{flushleft}

\end{frame}%%%%%%%%%%%%%%%%%%%%%%%%%%%%%%%%%%%%%%%%%%%%%%%%%%%%%%%%%%%%%%


\begin{frame}%%%%%%%%%%%%%%%%%%%%%%%%%%%%%%%%%%%%%%%%%%%%%%%%%%%%%%%%%%%%
  \frametitle{Box Composition}

  \begin{equation*}
   \raisebox{-4ex}{\scalebox{0.75}{\input{\figpath/comp_fct.pstex_t}}}
   \hspace{2em}
   \begin{aligned}[c]
    \iosqcmp{(\iosqcns{\ioout}{\iaiosq{1}})}{\iaiosq{2}}
    &= \iosqcns{\ioout}{(\iosqcmp{\iaiosq{1}}{\iaiosq{2}})}
    \\     
    \iosqcmp{(\iosqcns{\ioin}{\iaiosq{1}})}{(\iosqcns{\ioout}{\iaiosq{2}})}
    &= \iosqcmp{\iaiosq{1}}{\iaiosq{2}}
    \\
    \iosqcmp{(\iosqcns{\ioin}{\iaiosq{1}})}{(\iosqcns{\ioin}{\iaiosq{2}})} 
    &= \iosqcns{\ioin}{(\iosqcmp{(\iosqcns{\ioin}{\iaiosq{1}})}{\iaiosq{2}})}
  \end{aligned}
  \end{equation*}

\vspace*{1.5ex}
%
\begin{proposition}
  %
\begin{itemize}%[<+->]
  \item $\siosqbhv{\iosqcmp{\iaiosq{1}}{\iaiosq{2}}} 
    = \siosqbhv{\iaiosq{1}} \circ \siosqbhv{\iaiosq{2}}$
  \item associative
  \item preserves rationality %
  %$\iosqcmp{\iaiosq{1}}{\iaiosq{2}}\in\iosqrat$ 
  %whenever $\iaiosq{1},\iaiosq{2}\in\iosqrat$.
  \item 
    rat.~rep.\ of $\iosqcmp{\iaiosq{1}}{\iaiosq{2}}$
    can be computed
    from rat.~rep.'s of $\iaiosq{1}$ and $\iaiosq{2}$
\end{itemize}
  %
\end{proposition}


%% \vspace*{-0.5ex}
%% \renewcommand{\siosqcmp}{\chocolate{\boldsymbol{\cdot}}}
%% %
%% \begin{definition}\label{def:iosqcmp}\normalfont
%%   \emph{Composition} 
%%    $ \siosqcmp : \iosq\times\iosq \to \iosq $,
%%    $ \pair{\aiosq}{\biosq} \mapsto \iosqcmp{\aiosq}{\biosq} $  
%%   \emph{of \ioseqs}\\
%%   is corecursively defined by:  
%%   % 
%%   \begin{align*}
%%     \iosqcmp{(\iosqcns{\ioout}{\aiosq})}{\biosq}
%%     &= \iosqcns{\ioout}{(\iosqcmp{\aiosq}{\biosq})}
%%     \\     
%%     \iosqcmp{(\iosqcns{\ioin}{\aiosq})}{(\iosqcns{\ioout}{\biosq})}
%%     &= \iosqcmp{\aiosq}{\biosq}
%%     \\
%%     \iosqcmp{(\iosqcns{\ioin}{\aiosq})}{(\iosqcns{\ioin}{\biosq})} 
%%     &= \iosqcns{\ioin}{(\iosqcmp{(\iosqcns{\ioin}{\aiosq})}{\biosq})}
%%   \end{align*}
%%   %
%% \end{definition}
%% 
%% \pause
%% 
%% \vspace*{-0.5ex}
%% %
%% \begin{lemma}
%% \begin{itemize}[<+->]
%%     %
%%   \item $\siosqbhv{\iosqcmp{\aiosq}{\biosq}} = \siosqbhv{\aiosq} \circ \siosqbhv{\biosq}$.
%%     %
%%   \item Composition is associative.
%%     %
%%   \item Composition preserves rationality: $\iosqcmp{\aiosq}{\biosq} \in\iosqrat$ if $\aiosq,\biosq\in\iosqrat$.
%%     %
%%   \item On rational representations of rational \ioseqs,
%%     composition can be computed effectively.
%%     %% A rational representation of the composition of sequences
%%     %% with rational representations can be computed in finite time.
%%     %
%% \end{itemize}
%% \end{lemma}
%% 
%% \renewcommand{\siosqcmp}{{\cdot}}

\end{frame}%%%%%%%%%%%%%%%%%%%%%%%%%%%%%%%%%%%%%%%%%%%%%%%%%%%%%%%%%%%%%%


\begin{frame}%%%%%%%%%%%%%%%%%%%%%%%%%%%%%%%%%%%%%%%%%%%%%%%%%%%%%%%%%%%%
  \frametitle{Least Fixed Point of Box Composition}


  \vspace*{-0.5ex}
  %
  \renewcommand{\siosqfix}{\chocolate{\msf{fix}}}
    %
    \newcommand{\bsgn}{$\iosqbhv{\aiosq}{n}$}
    \newcommand{\exBa}{\iosqcns{\ioout}{\iosqcns{\ioout}{\nix}}}
    \newcommand{\exBc}{\iosqcyc{\iosqcns{\ioin}{\iosqcns{\ioout}{\iosqcns{\ioin}{\iosqcns{\ioout}{\iosqcns{\ioin}{\nix}}}}}}}
    \newcommand{\exB}{\exBa \exBc}
  \begin{center}
    %\scalebox{0.65}
    \fbox{\hspace{.5em}\input{\figpath/zz.pstex_t}\hspace{-.5em}}
    \\
    \mbox{}\\
    Graph of the production function $\siosqbhv{\aiosq}$
    for $\aiosq = \exB$ \\
    with least fixed point $\iosqfix{\aiosq} = 6$ 
    as indicated.
  \end{center}\pause{}

\vspace*{-0.75ex}
%
\begin{lemma}
  %
  \begin{itemize}
    %
    \item Given a rational representation $\pair{\alst}{\clst}$ 
      of $\aiosq\in\iosqrat$,
      its least fixed point\/ $\iosqfix{\aiosq}$
      can be computed in finite time.
      %
  \end{itemize}
  %
\end{lemma}
  %
  \renewcommand{\siosqfix}{\msf{fix}}


\end{frame}%%%%%%%%%%%%%%%%%%%%%%%%%%%%%%%%%%%%%%%%%%%%%%%%%%%%%%%%%%%%%%



%------------------------------------------------------------------------
\subsection{Net Reduction}
%------------------------------------------------------------------------
 

\begin{frame}%%%%%%%%%%%%%%%%%%%%%%%%%%%%%%%%%%%%%%%%%%%%%%%%%%%%%%%%%%%%
  \frametitle{From Nets to Sources}

\renewcommand{\scominm}{\chocolate{\msf{min}}}
\renewcommand{\ssiosqbhv}{\chocolate{\beta}}
\renewcommand{\siosqfix}{\chocolate{\msf{fix}}}
  %
  \vspace*{-0.5ex}
  %
\begin{definition}\label{def:coltrs}\normalfont
  %
\setcounter{equation}{0} 
\renewcommand{\theequation}{$\colrul${\arabic{equation}}}
  %
  \emph{Net reduction relation \magenta{$\scolred$}
        on closed pebbleflow nets}:
  %
  \vspace*{-0.5ex}
  %
  \begin{align*}
    \trmpeb{\atrm}
    &\red
    \trmbox{(\iosqcns{\ioout}{\iosqcyc{\iosqcns{\ioin}{\iosqcns{\ioout}{\nix}}}})}{\atrm}
    &
    %% \label{coltrs:peb}
    \\[-0.1ex]
    \trmbox{\aiosq}{\trmbox{\biosq}{\atrm}}
    &\red
      %% \trmbox{(\chocolate{\iosqcmp{\black{\aiosq}}{\black{\biosq}}})}{\atrm}
    \trmbox{\aiosq\mathrel{\chocolate{\boldsymbol{\cdot}}}\biosq}{\atrm}
    &
    %% \label{coltrs:boxbox}
    \\[-0.1ex]
    \trmbox{\aiosq}{\trmmeet{\atrm_1}{\atrm_2}}
    &\red
    \trmmeet{\trmbox{\aiosq}{\atrm_1}}{\trmbox{\aiosq}{\atrm_2}}
    &
    %% \label{coltrs:boxmeet}
    \\[-0.1ex]
    \trmrec{\anam}{\trmmeet{\atrm_1}{\atrm_2}}
    &\red
    \trmmeet{\trmrec{\anam}{\atrm_1}}{\trmrec{\anam}{\atrm_2}}
    &
    %% \label{coltrs:recmeet}
    \\[-0.1ex]
    \trmrec{\anam}{\atrm}
    &\red
    \atrm
    &\text{if $\anam\not\in\trmfv{\atrm}$}
    %% \label{coltrs:recrm}
    \\[-0.1ex]
    \trmrec{\anam}{\trmbox{\aiosq}{\anam}}
    &\red
    \trmsrc{ {{\iosqfix{{\aiosq}}}}}
    &
    %% \label{coltrs:recbox}
    \\[-0.1ex]
    \trmmeet{\trmsrc{ {k_1}}}{\trmsrc{ {k_2}}}
    &\red
    \trmsrc{ {\cominm{k_1}{k_2}}}
    &
    %% \label{coltrs:meetsrc}
    \\[-0.1ex]
    \trmbox{\aiosq}{\trmsrc{ {k}}}
    &\red
    \trmsrc{ {\iosqbhv{\aiosq}{k}}}
    &
    %% \label{coltrs:boxsrc}
    \\[-0.1ex]
    \trmrec{\anam}{\trmvar{\anam}}
    &\red
    \trmsrc{\conatzer}
    &
    %% \label{coltrs:recvar}
    %
  \end{align*}
  %
  \vspace*{-0.5ex}%
  %
  for all $\aiosq, \biosq \in \iosq$ and $k,k_1, k_2 \in \conat$.
  %
\end{definition}
%
\renewcommand{\ssiosqbhv}{\beta}
\renewcommand{\scominm}{\msf{min}}
\renewcommand{\siosqfix}{\msf{fix}}

\end{frame}%%%%%%%%%%%%%%%%%%%%%%%%%%%%%%%%%%%%%%%%%%%%%%%%%%%%%%%%%%%%%%

 
\begin{frame}%%%%%%%%%%%%%%%%%%%%%%%%%%%%%%%%%%%%%%%%%%%%%%%%%%%%%%%%%%%%
  \frametitle{Properties of Net Reduction}

\begin{theorem}
  \begin{itemize}%[<+->]
    %
    \item \darkcyan{$\colred$} is production preserving:
      \[
        \atrm \mathrel{\darkcyan{\colred}} \atrm' 
        \;\;\;\Longrightarrow\;\;\;
        \netprd{\atrm} = \netprd{\atrm'} \; .
      \]
      %
      \vspace*{-1ex}
      % 
    \item \darkcyan{$\scolred$} is confluent and terminating. 
      %
    \item Every closed net normalises to a source, 
      its unique\\ \darkcyan{$\scolred$}-normal form.
      %
  \item For every rational net $\atrm$, the \darkcyan{$\scolred$}-normal form
      of $\atrm$ can be computed effectively. 
    %% 
    %% There is an algorithm that, 
    %% given $\atrm\in\trmrat$ and rational representations 
    %% of all sequences $\aiosq\in\iosqrat$,
    %% computes the $\scolred$-normal form of $\atrm$.
    %
  \end{itemize}
\end{theorem}

\end{frame}%%%%%%%%%%%%%%%%%%%%%%%%%%%%%%%%%%%%%%%%%%%%%%%%%%%%%%%%%%%%%


%------------------------------------------------------------------------
\subsection{Main Result. Examples. J\"{o}rg's Tool.}
%------------------------------------------------------------------------
 
\begin{frame}%%%%%%%%%%%%%%%%%%%%%%%%%%%%%%%%%%%%%%%%%%%%%%%%%%%%%%%%%%%%
  \frametitle{Deciding Productivity for pure SCS{s}}

\begin{theorem}
  Productivity for pure SCSs is decidable.%\pause{}
\end{theorem}


\begin{proof}
  %
  A decision algorithm for productivity of an SCS~$\atrs$:
  %
  \begin{enumerate}[<+->]
    %
    \item  Translate $\rootsc$ to the rational net $\trnsl{\rootsc}$.
      %
    \item Reduce $\trnsl{\rootsc}$ to a source $\trmsrc{\darkcyan{n}}$.\\
      % 
      (Note that $\terbhv{\atrs}{\rootsc} 
               = \trmbhv{\trnsl{\rootsc}} = \darkcyan{n}$.)
      %
    \item If $\darkcyan{n} = \darkcyan{\boldsymbol{\conattop}}$, 
      then output: ``\alert{$\atrs$ is productive}'';\\
      else, $\darkcyan{n}\in\nat$, 
      output: ``\alert{$\atrs$ is not productive}, 
      it only produces\\ \hspace*{\fill}
      $\darkcyan{n}$ data-elements for $\rootsc$''.
      %
  \end{enumerate}
  %
\end{proof}

\end{frame}%%%%%%%%%%%%%%%%%%%%%%%%%%%%%%%%%%%%%%%%%%%%%%%%%%%%%%%%%%%%%%



%------------------------------------------------------------------------
%\subsection{Examples, and our Tools}
%------------------------------------------------------------------------



%%%%%%%%%%%%%%%%%%%%%%%%%%%%%%%%%%%%%%%%%%%%%%%%%%%%%%%%%%%%%%%%%%%%%%%%


\begin{frame}%%%%%%%%%%%%%%%%%%%%%%%%%%%%%%%%%%%%%%%%%%%%%%%%%%%%%%%%%%%
  \frametitle{Net Reduction Tool: Computing Net Production}

  \vspace*{0.5ex}
  %
  \begin{itemize}
    %
    \item  \emph{Translation and reduction tool} (Haskell-based)
      by J\"{o}rg Endrullis.\\[0.5ex]
      %
      Use it at:
      \href{http://infinity.few.vu.nl/productivity}
           {\emph{\tt http://infinity.few.vu.nl/productivity}} 
      %
  \end{itemize}

    
  %\vspace*{-0.5ex}
  %
  \begin{align*}
    %
    \href{http://infinity.few.vu.nl/productivity/\#thuemorse_alternative}{\morse} 
      &= \strcns{\datsymb{0}}{\strzip{\strinv{\morse}}{\strtl{\morse}}}
    \\
    \href{http://infinity.few.vu.nl/productivity/\#thuemorse_alternative}{\terprd{\atrs}{\morse}}
      &= \netprd{\trnsl{\morse}} 
       = \boldsymbol{\conattop}
    \\[2.5ex]
    %
    \href{http://infinity.few.vu.nl/productivity/\#J}{\joerg} 
      &= \strcns{\datsymb{0}}{\strcns{\datsymb{1}}{\strev{\joerg}}}
    \\
    \href{http://infinity.few.vu.nl/productivity/\#J}{\terprd{\atrs}{\joerg}}
      & = \netprd{\trnsl{\joerg}}
        = 4
    \\[2.5ex]
    %
    \href{http://infinity.few.vu.nl/productivity/\#D}{\dimitri} 
      &= \strcns{\datsymb{0}}{\strcns{\datsymb{1}}{\strcns{\datsymb{0}}{
                    \strzip{\stradd{\strtl{\dimitri}}{\strtl{\strtl{\dimitri}}}}
                           {\strev{\strtl{\dimitri}}} }}}
    \\
    \href{http://infinity.few.vu.nl/productivity/\#D}{\terprd{\atrs}{\dimitri}}
      &= \netprd{\trnsl{\dimitri}}
       = \boldsymbol{\conattop}
      %
  \end{align*}

\end{frame}%%%%%%%%%%%%%%%%%%%%%%%%%%%%%%%%%%%%%%%%%%%%%%%%%%%%%%%%%%%%%


%-----------------------------------------------------------------------
%-----------------------------------------------------------------------
\section{Summary and Extensions}
%-----------------------------------------------------------------------
%-----------------------------------------------------------------------

%------------------------------------------------------------------------
%\subsection{Conclusion and Ongoing Research}
%------------------------------------------------------------------------


\begin{frame}%%%%%%%%%%%%%%%%%%%%%%%%%%%%%%%%%%%%%%%%%%%%%%%%%%%%%%%%%%%%
  \frametitle{Summary and Extensions}


\begin{itemize}%[<+->]

  \item
  \chocolate{Previous Approaches}: 
  \emph{sufficient conditions} for productivity,\\
  not automatable or only for a limited subclass

  \item
  \chocolate{Our Contribution}:
  \emph{decision algorithm} for a rich class of SCSs,\\
     only SFS part is restricted

  \item
  \chocolate{Recent Results}:
    %
    \begin{itemize}
      %
      \item We increase the applicability of the pebbleflow method
        by translating to nets that bound the production of 
        SCS{s} from below and above.
      \item
        We obtain, for some classes of SCSs, \emph{computable sufficient
        conditions} for productivity, and for non-productivity. 
        %
      \item For a class of SFSs with \emph{pattern matching on data}
        we give a computable, data-obliviously optimal,
        sufficient condition for productivity.
        %
   \end{itemize}
   %
\end{itemize}

\end{frame}%%%%%%%%%%%%%%%%%%%%%%%%%%%%%%%%%%%%%%%%%%%%%%%%%%%%%%%%%%%%%%  


\begin{frame}%%%%%%%%%%%%%%%%%%%%%%%%%%%%%%%%%%%%%%%%%%%%%%%%%%%%%%%%%%%%
  \frametitle{Application to Larger Classes}


  \begin{center}
    %
    \scalebox{.7}{\includegraphics{../../fossacs/submitted/figs/pstricks/bones}}
    %
  \end{center}

\end{frame}%%%%%%%%%%%%%%%%%%%%%%%%%%%%%%%%%%%%%%%%%%%%%%%%%%%%%%%%%%%%%%  


\begin{frame}%%%%%%%%%%%%%%%%%%%%%%%%%%%%%%%%%%%%%%%%%%%%%%%%%%%%%%%%%%%%
  \frametitle{{\bf PM} SCS}


\begin{example}
  %
  \vspace*{2.5ex}
  %
\begin{center}
\scalebox{1.1}{
\renewcommand{\arraystretch}{1.25}
\begin{tabular}{|c|l|}
  \hline
  $\rootsc \red 
      \strcns{\datsymb{0}}{\strcns{\datsymb{1}}{\funap{\strfun{f}}{\rootsc}}}$
  & SCS-layer\\
  \hline
  $\funap{\strfun{f}}{\strcns{\datsymb{0}}{\strcns{x}{\astr}}} \red
     \strcns{\datsymb{1}}{\strcns{\datsymb{0}}{\strcns{x}{\funap{\strfun{f}}{\astr}}}} $ 
  & \\[-0.0ex]
  $\funap{\strfun{f}}{\strcns{1}{\astr}} \red
     \strcns{\datsymb{0}}{\funap{\strfun{f}}{\astr}} $ 
  & SFS-layer\\[-0.0ex]
  \hline
  \phantom{
    $\funap{\datsymb{i}}{\datsymb{0}} \red \datsymb{1}\quad\quad \funap{\datsymb{i}}{\datsymb{1}} \red \datsymb{0}$
          }
  & data-layer \\
  \hline
\end{tabular}
               }
\end{center}
  %
  \mbox{}\\[0.5ex]
  %
\end{example}

\vspace*{1ex}
%
\begin{flushleft}
  \scalebox{1.25}{This {\bf PM} SCS is \emph{productive}}.
\end{flushleft}
 
\end{frame}%%%%%%%%%%%%%%%%%%%%%%%%%%%%%%%%%%%%%%%%%%%%%%%%%%%%%%%%%%%%%%


\begin{frame}%%%%%%%%%%%%%%%%%%%%%%%%%%%%%%%%%%%%%%%%%%%%%%%%%%%%%%%%%%%%
  \frametitle{Our Papers and Tools.}

  \vspace*{0.5ex}
  %
  Please visit \href{http://infinity.few.vu.nl/productivity}%
         {\emph{\tt http://infinity.few.vu.nl/productivity}}\\
  to find: 
  %
  \vspace*{1.5ex}
  %
  \begin{itemize}
    %
    \item \emph{Productivity of Stream Definitions},
      Proceedings of FCT~2007, LNCS~4637, pages~274--287, 2007;
      %
      \vspace*{0.75ex}
      %
    \item \emph{Productivity of Stream Definitions},
      Technical Report;
      %
      \vspace*{0.75ex}
      %
    \item \emph{Sufficiently productive? Or not productive enough?},
      submitted;
      %
      \vspace*{0.75ex}
      %
    \item and to access and use our tools. 
      %
  \end{itemize}

\end{frame}%%%%%%%%%%%%%%%%%%%%%%%%%%%%%%%%%%%%%%%%%%%%%%%%%%%%%%%%%%%%%


\begin{frame}%%%%%%%%%%%%%%%%%%%%%%%%%%%%%%%%%%%%%%%%%%%%%%%%%%%%%%%%%%%%
  \frametitle{}

   \vspace*{\fill}
   %
   \hspace{\fill}%
   \scalebox{1.6}{\forestgreen{\bf Thanks for your attention!}}%
   \hspace*{\fill}\mbox{}
   %
   \label{lastslide}
   %
   \vspace*{\fill}

 %% \only<7>{
 %% \begin{center}
 %% \large{Thank you!}
 %% \end{center}
 %% }

\end{frame}%%%%%%%%%%%%%%%%%%%%%%%%%%%%%%%%%%%%%%%%%%%%%%%%%%%%%%%%%%%%%


\end{document}
%    \item 
 %     Better understanding of \emph{related work} 
 %     (Telford--Turner, Hughes--Pareto--Sabry, Abel, Buchholz, \ldots).
%    \item
%      What is the \emph{exact class} of recursive stream programs 
%      that we can decide productivity of?

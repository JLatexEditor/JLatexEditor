\documentclass[10pt,handout]{beamer}

\input{packages}
\input{dosp.def}


\mode<presentation>
{
  \usetheme{Berlin}  %% \usetheme{default} 
  \setbeamercovered{transparent}
  \usecolortheme{seahorse}
  \usecolortheme{rose} %% \usecolortheme{crane} % (oranje)
  \setbeamertemplate{navigation symbols}{}
\setbeamertemplate{itemize items}[triangle]
\setbeamertemplate{sections/subsections in toc}[sections numbered]
%\setbeamercolor{math text}{fg=black}
}

%------------------------------------------------------------------------


%%\usepackage[german]{babel}
% oder was auch immer

\usepackage[latin1]{inputenc}
% oder was auch immer

\usepackage{times}
\usepackage[T1]{fontenc}
% Oder was auch immer. Zu beachten ist, das Font und Encoding passen
% m�ssen. Falls T1 nicht funktioniert, kann man versuchen, die Zeile
% mit fontenc zu l�schen.


\title{Data-Oblivious Stream Productivity}

%\title[Watching Streams Grow]{\scalebox{0.99}%
%         {\hspace*{-1.5pt}Watching \hspace*{-1.5pt}Streams \hspace*{-1.5pt}Grow: 
%          \hspace*{-1.5pt}The \hspace*{-1.5pt}Pebbleflow \hspace*{-1.5pt}Method}}

\author[Endrullis, Grabmayer, Hendriks]{
  J\"{o}rg Endrullis         %\inst{1}
  \and {Clemens Grabmayer}   %\inst{2}
  \and Dimitri Hendriks      \\%\inst{1}
}

%% \institute{
%%   Universiteit Utrecht, 
%%   Vrije Universiteit
%%                               \\[2ex]
%% {\chocolate{\large NWO-Bricks Project Infinity}}  %{{ $\boldsymbol{\infty}$}}
%%           }
  
%% \institute{
%%   Department of Philosophy, \emph{Universiteit Utrecht}\\[0.5ex]
%%   Department of Computer Science, \emph{Vrije Universiteit}\\[0.5ex]
%%   Department of Computer Science, \emph{Radboud Universiteit}   
%%            }

\institute[PAM, CWI, May 7, 2008]{Vrije Universiteit~--~Universiteit Utrecht~--~Vrije Universiteit}

\date{
  {NWO Projects \chocolate{Infinity}
                and \chocolate{ProvCyc}}\\[2ex]
  \forestgreen{PAM}, \emph{CWI}\\[1ex] 
      May 7, 2008}


\subject{Informatik}
% Dies wird lediglich in den PDF Informationskatalog einf�gt. Kann gut
% weggelassen werden.


% Falls eine Logodatei namens "university-logo-filename.xxx" vorhanden
% ist, wobei xxx ein von latex bzw. pdflatex lesbares Graphikformat
% ist, so kann man wie folgt ein Logo einf�gen:

% \pgfdeclareimage[height=0.5cm]{university-logo}{university-logo-filename}
% \logo{\pgfuseimage{university-logo}}


%% % Folgendes sollte gel�scht werden, wenn man nicht am Anfang jedes
%% % Unterabschnitts die Gliederung nochmal sehen m�chte.
   \AtBeginSection[]
   {
   \begin{frame}<beamer>
     \frametitle{Overview}
     \tableofcontents[currentsection] %[...,currentsubsection]
   \end{frame}
    }


% Falls Aufz�hlungen immer schrittweise gezeigt werden sollen, kann
% folgendes Kommando benutzt werden:

%\beamerdefaultoverlayspecification{<+->}


%----------------------------------------------------------------------------------


\begin{comment}
  %
  % abstract
  %
  I will be reporting on joint work with Joerg Endrullis and Dimitri 
  Hendriks: we explore the boundaries of an approach we developed in an
  earlier collaboration with Ariya Isihara and Jan Willem Klop.

  We are interested in finite specifications of infinite streams of data
  that are based on orthogonal rewrite rules. A stream specification is 
  called `productive' if it can be evaluated continually with a unique 
  data stream being obtained as the limit. Whereas productivity is 
  undecidable in general, for restricted formats computable sufficient
  conditions can be found, and in some cases even decidability can be
  established.

  The usual analysis of the productivity problem, which we also adopt, 
  disregards the identity of data, thus leading to approaches that we call
  "data-oblivious". We found a method that is provably optimal among all 
  such data-oblivious approaches. This means that in order to improve on 
  our algorithm one has to proceed in a data-aware fashion.

  For more information about our work on productivity of stream 
  definitions, please visit http://infinity.few.vu.nl/productivity/ .
  %
\end{comment}

%----------------------------------------------------------------

\begin{document}

%\renewcommand{\pause}{}



\begin{frame}%%%%%%%%%%%%%%%%%%%%%%%%%%%%%%%%%%%%%%%%%%%%%%%%%%%%%%%%%%%%

\titlepage

\end{frame}%%%%%%%%%%%%%%%%%%%%%%%%%%%%%%%%%%%%%%%%%%%%%%%%%%%%%%%%%%%%%%

%\setcounter{tocdepth}{1}

\begin{frame}%%%%%%%%%%%%%%%%%%%%%%%%%%%%%%%%%%%%%%%%%%%%%%%%%%%%%%%%%%%%
  \frametitle{Outline}

\tableofcontents%[pausesections]

\end{frame}%%%%%%%%%%%%%%%%%%%%%%%%%%%%%%%%%%%%%%%%%%%%%%%%%%%%%%%%%%%%%%

%------------------------------------------------------------------------
%------------------------------------------------------------------------
\section{Introduction}
%------------------------------------------------------------------------
%------------------------------------------------------------------------

%------------------------------------------------------------------------
\subsection{Productivity. Previous Approaches.}
%------------------------------------------------------------------------


\begin{frame}%%%%%%%%%%%%%%%%%%%%%%%%%%%%%%%%%%%%%%%%%%%%%%%%%%%%%%%%%%%%
  \frametitle{Productivity}

\begin{itemize}
  %
  \item `Productivity' first used by \emph{Dijkstra (1980)}.
    %
    \vspace*{0.5ex}
    %
  \item Slogan: 
    {\em For programming with infinite structures, productivity is what
      termination is for programming with finite structures.}
      %
    \vspace*{0.5ex}
    %
  \item Productivity captures the notion of \emph{unlimited progress},
    of `working' programs, producing defined values indefinitely. 
    %
\end{itemize}


\vspace*{0.5ex}
%
Related questions:
%
\begin{itemize}
  %
  \item When do we accept an infinite 
    object defined in terms of itself?
    %
    \vspace*{0.5ex}
    %
  \item When does a finite set of equations
    \emph{constructively} define a \emph{unique} infinite object?
    %
\end{itemize}

\end{frame}%%%%%%%%%%%%%%%%%%%%%%%%%%%%%%%%%%%%%%%%%%%%%%%%%%%%%%%%%%%%%%


\begin{frame}%%%%%%%%%%%%%%%%%%%%%%%%%%%%%%%%%%%%%%%%%%%%%%%%%%%%%%%%%%%%
  \frametitle{Productivity of Stream Specifications}

\begin{itemize}
  %
  \item $\str{A} \defdby \{ \astr \where \astr \funin {\nat\to A} \}$
    the set $\str{A}$ of \emph{streams over $A$}
    %
  \item `$\sstrcns$'\ is the \emph{stream constructor} symbol:
    \;\(\strcns{\datsymb{a}}{\astr}\)\;
    denotes the result of prepending 
    \;$\datsymb{a}\in A$\; to \;$\astr\in\str{A}$
    %
  \item  A recursive stream specification 
    \[
      \astrcon = \ldots\, {\astrcon} \,\ldots
    \]
    is \emph{productive} if the process of continually evaluating $\astrcon$ 
    
    results in an infinite \emph{constructor normal form}:
    \[
      \astrcon
      \infred
      \strcns{\datsymb{a_0}}{\strcns{\datsymb{a_1}}{\strcns{\datsymb{a_2}}{\ldots}}}
    \]
  % \item A stream definition is \emph{productive}
  %  if it can be evaluated continually with a uniquely
  %  determined stream is obtained as the limit.
    %
  \item Productivity is \emph{undecidable} in general
    (in fact \emph{$\Pi^0_2$-complete}). 
    %
  \item But for \forestgreen{restricted formats} 
    \emph{computable sufficient conditions}\\
    or \emph{decidability} can be obtained.
    %
\end{itemize}

\end{frame}%%%%%%%%%%%%%%%%%%%%%%%%%%%%%%%%%%%%%%%%%%%%%%%%%%%%%%%%%%%%%%


\begin{frame}%%%%%%%%%%%%%%%%%%%%%%%%%%%%%%%%%%%%%%%%%%%%%%%%%%%%%%%%%%%%
  \frametitle{Examples}

\begin{example}
  \begin{align*}
    \uncover<1,5>{
    \strread{\strcns{x}{\astr}}
               }
      &
    \uncover<1,5>{
      = \strcns{x}{\strread{\astr}}
                } 
      & \\
    \uncover<2,5>{
    \strfread{\strcns{x}{\strcns{y}{\astr}}}
                }
      &
    \uncover<2,5>{
      = \strcns{x}{\strcns{y}{\strfread{\astr}}}
               } 
      & \\
    \uncover<1,5>{
    \strcon{fives}
                } 
      &
    \uncover<1,5>{= \strcns{\datf{5}}{\strread{\strcon{fives}}}} 
      & & 
    \uncover<1,5>{\text{productive}} \\
    \uncover<2,5>{
    \strcon{fives'} 
                }
      & \uncover<2,5>{= \strcns{\datf{5}}{\strfread{\strcon{fives'}}}} 
      & & \uncover<2,5>{\text{not productive}} \\
    \uncover<3,5>{ 
    \strzipb{\strcns{x}{\astr}}{\strcns{y}{\bstr}}
                } 
      & \uncover<3,5>{= \strcns{x}{\strcns{y}{\strzipb{\astr}{\bstr}}}}
      & \\
    \uncover<4,5>{
    \strzipa{\strcns{x}{\astr}}{\bstr}
               } 
      & \uncover<4,5>{= \strcns{x}{\strzipa{\bstr}{\astr}}} 
      & \\
    \uncover<3,5>{
    \strcon{sevens} 
                }
      & \uncover<3,5>{
        = \strcns{\datf{7}}{\strzipb{\strcon{sevens}}{\strtl{\strcon{sevens}}}}
                    }
      & & \uncover<3,5>{\text{not productive}} \\ 
    \uncover<4,5>{
    \strcon{sevens'} 
                }
      &\uncover<4,5>{=
       \strcns{\datf{7}}{\strzipa{\strcon{sevens'}}{\strtl{\strcon{sevens'}}}}
                   }
      & & \uncover<4,5>{\text{productive}} 
  \end{align*}
\end{example}

\end{frame}%%%%%%%%%%%%%%%%%%%%%%%%%%%%%%%%%%%%%%%%%%%%%%%%%%%%%%%%%%%%%%


\begin{frame}%%%%%%%%%%%%%%%%%%%%%%%%%%%%%%%%%%%%%%%%%%%%%%%%%%%%%%%%%%%%
  \frametitle{Productivity Recognition: Previous Approaches}


\begin{itemize}
  %
  \item \emph{Wadge (1981)}: `cyclic sum test'
    (\forestgreen{limited, computable criterion}).
    %
  \item \emph{Sijtsma (1989)}: mathematical theory of productivity
    based on `production moduli' 
    (\forestgreen{mathem., not directly computable criteria}).\pause{}
    %
  \item \emph{Coquand (1994)}: `guardedness' as a syntactic criterion
    for productivity 
    (\forestgreen{automatable}, but \forestgreen{restrictive criterion}).
    %
  \item \emph{Telford and Turner (1997)}: extend the notion of
    guardedness by a method in the flavour of Wadge.
    %
  \item \emph{Hughes, Pareto, and Sabry (1996)}: introduce a type system
    for proving productivity (\forestgreen{automatable criterion}).\pause{}
    %
  \item \emph{Buchholz (2004)}: type system for proving productivity,
    two forms:
    %
    \vspace*{-1.5ex}
    %
    \begin{itemize}
      %
      \item using unrestricted production moduli 
        (\forestgreen{general, not automatable});
        %
      \item a decidable subsystem with limited moduli
        (\forestgreen{automable criterion}, 
         handles all examples of Telford \&Turner).\pause{}
        %
    \end{itemize}
    %
\end{itemize}

\end{frame}%%%%%%%%%%%%%%%%%%%%%%%%%%%%%%%%%%%%%%%%%%%%%%%%%%%%%%%%%%%%%%


\begin{frame}%%%%%%%%%%%%%%%%%%%%%%%%%%%%%%%%%%%%%%%%%%%%%%%%%%%%%%%%%%%%
  \frametitle{Our First Paper: Productivity Decision}

\emph{Productivity of Stream Definitions}
(Endrullis, Grabmayer, Hendriks, Isihara, Klop), FCT'07:

\vspace*{0.5ex}
%
\begin{itemize}
  %
  \item A \chocolate{decision algorithm for productivity}
    on the large and natural class of \emph{pure stream spec's}.
    %
    \vspace*{0.85ex}
    %
  \item `Large class': The stream functions allowed 
    in pure stream spec's contain all \emph{automatic sequences}
    (Allouche, Shallit). 
    %
    \vspace*{0.85ex}
    %
  \item Idea behind the decision algorithm:
    %
    \begin{itemize}
        %
        \vspace*{0.25ex}
        %
      \item The process of evaluation of a pure stream spec can
        be modelled by dataflow of \emph{pebbles} in a finite
        \emph{pebbleflow net}.
        %
        \vspace*{0.25ex}
        %
      \item The production of a pebbleflow net associated with
        a pure stream spec (amount of pebbles the net can produce
        at its output port) can be calculated by reducing nets
        to trivial nets.
        %
    \end{itemize}
    %
\end{itemize}

\end{frame}%%%%%%%%%%%%%%%%%%%%%%%%%%%%%%%%%%%%%%%%%%%%%%%%%%%%%%%%%%%%%%

%------------------------------------------------------------------------
\subsection{First Paper. New Results.}
%------------------------------------------------------------------------



\begin{frame}%%%%%%%%%%%%%%%%%%%%%%%%%%%%%%%%%%%%%%%%%%%%%%%%%%%%%%%%%%%%
  \frametitle{Main New Results: Productivity Recognition/Decision}

\begin{itemize}
  %
  \item All previous approaches use a `quantitative' analysis that
    abstracts away from concrete values of stream elements.\\
    We formalise this by \emph{data-oblivious rewriting}. 
    %
  \item We introduce the notion of \emph{data-oblivious productivity}.
    %
  \item We identify two syntactical classes of stream spec's:
    \emph{flat} and (general) \emph{pure} spec's.
    % 
  \item For \emph{flat} stream spec's we obtain a 
    \emph{decision method for data-oblivious productivity},
    yielding a
    \chocolate{computable, data-obliviously optimal criterion
    for productivity}. 
    %
  \item For \emph{pure} stream spec's we obtain
    a \chocolate{decision method for productivity}. 
    %
\end{itemize}

\end{frame}%%%%%%%%%%%%%%%%%%%%%%%%%%%%%%%%%%%%%%%%%%%%%%%%%%%%%%%%%%%%%%


\begin{frame}%%%%%%%%%%%%%%%%%%%%%%%%%%%%%%%%%%%%%%%%%%%%%%%%%%%%%%%%%%%%
  \frametitle{Map of Stream Specifications}  

    
\begin{center}
  %
  \scalebox{1.15}{ 
    \includegraphics{figs/pstricks/pnp}
              }
  %
\end{center}

\end{frame}%%%%%%%%%%%%%%%%%%%%%%%%%%%%%%%%%%%%%%%%%%%%%%%%%%%%%%%%%%%%%%



%------------------------------------------------------------------------
%------------------------------------------------------------------------
\section{Stream Specifications}
%------------------------------------------------------------------------
%------------------------------------------------------------------------

%------------------------------------------------------------------------
\subsection{Motivating Examples}
%------------------------------------------------------------------------

\begin{frame}%%%%%%%%%%%%%%%%%%%%%%%%%%%%%%%%%%%%%%%%%%%%%%%%%%%%%%%%%%%%
  \frametitle{Stream Specification}

\begin{example}
  %
  \vspace*{0.5ex}
  %
\begin{center}
\scalebox{1}{
\renewcommand{\arraystretch}{1.3}
\begin{tabular}{r@{\hspace*{1.5ex}}cr}
  \hline
  &
  $\morse \red \strcns{\datsymb{0}}{\strzip{\strinv{\morse}}{\strtl{\morse}}}$ 
  & {\it stream layer}\\
  \hline
  &
  $\strtl{\strcns{x}{\astr}} \red \astr$ & \\[-0.0ex]
  &
  $\strzip{\strcns{x}{\astr}}{\bstr} \red \strcns{x}{\strzip{\bstr}{\astr}}$ 
  & {\it function layer} \\[-0.0ex]
  &
  $\strinv{\strcns{x}{\astr}} \red \strcns{\funap{\datsymb{i}}{x}}{\strinv{\astr}}$ & \\
  \hline
  &
  $\funap{\datsymb{i}}{\datsymb{0}} \red \datsymb{1} \quad\quad \funap{\datsymb{i}}{\datsymb{1}} \red \datsymb{0}$
  & {\it data layer} \\
  \hline
\end{tabular}
               }
\end{center}
  %
  \mbox{}\\[-0.5ex]
  %
\end{example}
 
\vspace*{-1.5ex}
%
\begin{flushleft}
  \scalebox{1}{This is a \emph{productive} stream definition
                 obtaining the \emph{Thue-Morse sequence}:}\\[1.5ex]
  \hspace*{\fill}
  \scalebox{1}{
  $
  \morse \infred \strcns{0}{\strcns{1}{\strcns{1}\strcns{0}{
                 \strcns{1}{\strcns{0}{\strcns{0}\strcns{1}{
                 \strcns{1}{\strcns{0}{\strcns{0}\strcns{1}{
                 \strcns{0}{\strcns{1}{\strcns{1}\strcns{0}{
                   \ldots}}}}}}}}}}}}
  $
                  }
\end{flushleft}

\end{frame}%%%%%%%%%%%%%%%%%%%%%%%%%%%%%%%%%%%%%%%%%%%%%%%%%%%%%%%%%%%%%%

\begin{comment}
\begin{frame}%%%%%%%%%%%%%%%%%%%%%%%%%%%%%%%%%%%%%%%%%%%%%%%%%%%%%%%%%%%%
  \frametitle{Stream Specifications}

\begin{example}
  %
  \vspace*{2.5ex}
  %
\begin{center}
\scalebox{1}{
\renewcommand{\arraystretch}{1}
\begin{tabular}{cr}
  \hline
  $\morse \red \strcns{\datsymb{0}}{\strzip{\strinv{\morse}}{\strtl{\morse}}}$ 
  & {\it stream layer}\\
  \hline
  $\strtl{\strcns{x}{\astr}} \red \astr$ & \\[-0.0ex]
  $\strzip{\strcns{x}{\astr}}{\bstr} \red \strcns{x}{\strzip{\bstr}{\astr}}$ 
  & {\it function layer} \\[-0.0ex]
  $\strinv{\strcns{x}{\astr}} \red \strcns{\funap{\datsymb{i}}{x}}{\strinv{\astr}}$ & \\
  \hline
  $\funap{\datsymb{i}}{\datsymb{0}} \red \datsymb{1} \quad\quad \funap{\datsymb{i}}{\datsymb{1}} \red \datsymb{0}$
  & {\it data layer} \\
  \hline
\end{tabular}
               }
\end{center}
  %
  \mbox{}\\[0.5ex]
  %
\end{example}
 
\begin{flushleft}
  \scalebox{1}{This is a \emph{productive} stream spec
               for the \emph{Thue-Morse sequence}:}\\[1.5ex]
  \hspace*{\fill}
  \scalebox{1}{
  $
  \morse \infred \strcns{0}{\strcns{1}{\strcns{1}\strcns{0}{
                 \strcns{1}{\strcns{0}{\strcns{0}\strcns{1}{
                 \strcns{1}{\strcns{0}{\strcns{0}\strcns{1}{
                 \strcns{0}{\strcns{1}{\strcns{1}\strcns{0}{
                   \ldots}}}}}}}}}}}}
  $
                  }
\end{flushleft}

\end{frame}%%%%%%%%%%%%%%%%%%%%%%%%%%%%%%%%%%%%%%%%%%%%%%%%%%%%%%%%%%%%%%
\end{comment}

\begin{frame}%%%%%%%%%%%%%%%%%%%%%%%%%%%%%%%%%%%%%%%%%%%%%%%%%%%%%%%%%%%%
  \frametitle{Stream Specification}


\begin{example}
  %
  \vspace*{1ex}
  %
\begin{center}
\scalebox{1}{
\renewcommand{\arraystretch}{1.25}
\begin{tabular}{r@{\hspace{1.5ex}}cr}
  \hline
  &
  $\joerg 
      = \strcns{\datsymb{0}}{\strcns{\datsymb{1}}{\strev{\joerg}}}$
  & {\it stream layer}\\
  \hline
  &
  $\strev{\strcns{x}{\astr}} \red \strcns{x}{\strod{\astr}}$ & \\[-0.0ex]
  &
  $\strod{\strcns{x}{\astr}} \red \strev{\astr} $ 
  & {\it function layer}\\[-0.0ex]
  \hline
  &
  \phantom{
    $\funap{\datsymb{i}}{\datsymb{0}} \red \datsymb{1}\quad\quad \funap{\datsymb{i}}{\datsymb{1}} \red \datsymb{0}$
          }
  & {\it data layer}\\
  \hline
\end{tabular}
               }
\end{center}
  %
  \mbox{}\\[-0.5ex]
  %
\end{example}

\vspace*{1ex}
%
\begin{flushleft}
  \scalebox{1}{This stream definition is \emph{not productive}: \mbox{}
  $
  \joerg \infred \strcns{\datf{0}}{\strcns{\datf{1}}{\strcns{\datf{0}}\strcns{\datf{0}}{
                                           \sstrev^{\omega}}}}
  $
                  }
\end{flushleft}
 
\end{frame}%%%%%%%%%%%%%%%%%%%%%%%%%%%%%%%%%%%%%%%%%%%%%%%%%%%%%%%%%%%%%%


\begin{frame}%%%%%%%%%%%%%%%%%%%%%%%%%%%%%%%%%%%%%%%%%%%%%%%%%%%%%%%%%%%
  \frametitle{$\joerg %\red \strcns{\datf{0}}{\strcns{\datf{1}}{\strev{\joerg}}}
                      \infred 
                       \strcns{\datf{0}}{\strcns{\datf{1}}{\strcns{\datf{0}}\strcns{\datf{0}}{
                                  \sstrev^{\omega}}}}$}
  \vspace*{-3.75ex}
  %
  \begin{align*}
    %
    %\uncover<1->{
    \joerg  
      & \red 
        \strcns{\datf{0}}{\strcns{\datf{1}}{\strev{\joerg}}}  
    %             }
        %
                                                                  \\[0.2ex]
        %
    %\uncover<2->{
    \strev{\joerg} 
      & \red
        \strev{\strcns{\datf{0}}{\strcns{\datf{1}}{\strev{\joerg}}}}
                                                                    \\[-0.35ex]
      & \red
        \strcns{\datf{0}}{\strod{\strcns{\datf{1}}{\strev{\joerg}}}}
                                                                    \\[-0.35ex]
      & \red
        \strcns{\datf{0}}{\strev{\strev{\joerg}}} 
    %             }
        %
                                                                  \\[0.2ex]
        %
    %\uncover<3->{
    \strevsq{\joerg} \equiv
    \strev{\strev{\joerg}}
      & \mred 
        \strev{ \strcns{\datf{0}}{\strev{\strev{\joerg}}} }
                                                                    \\[-0.35ex]
      & \red
        \strcns{\datf{0}}{\strod{\strevsq{\joerg}}} 
    %             }
        %
                                                                  \\[0.2ex]
        %
    %\uncover<4->{
    \strod{\strevsq{\joerg}} 
      & \mred 
        \strod{\strcns{\datf{0}}{\strod{\strevsq{\joerg}}}}
                                                                    \\[-0.35ex]
      & \red
        \strev{\strod{\strevsq{\joerg}}} 
    %             }
        %
                                                                  \\[0.2ex]
        %
    %\uncover<5->{
    \strod{\strevsq{\joerg}}
      & \mred
        \strev{\strod{\strevsq{\joerg}}}
                                                                    \\[-0.35ex]
      & \mred
        \strevsq{\strod{\strevsq{\joerg}}}
                                                                    \\[-0.35ex]
      & \mred \ldots
        \mred \strevn{\strod{\strevsq{\joerg}}} 
        \mred \ldots
                                                                    \\[-0.35ex]
      & \infred 
        \sstrevom 
    %             }
        %
  \end{align*}

  \vspace*{-2ex}
  %\uncover<6->{
  Hence: $\joerg \infred 
                     \strcns{\datf{0}}{\strcns{\datf{1}}{\strcns{\datf{0}}\strcns{\datf{0}}{
                                           \sstrev^{\omega}}}}$.
  %            } 

\end{frame}%%%%%%%%%%%%%%%%%%%%%%%%%%%%%%%%%%%%%%%%%%%%%%%%%%%%%%%%%%%%%%
     

%------------------------------------------------------------------------
\subsection{Definitions and Classes}
%------------------------------------------------------------------------


\begin{frame}%%%%%%%%%%%%%%%%%%%%%%%%%%%%%%%%%%%%%%%%%%%%%%%%%%%%%%%%%%%%
  \frametitle{Stream TRS}

A \magenta{stream TRS} is a
%
\begin{itemize}
  %
  \item finite $\{\sortS,\sortD\}$-sorted,
     \forestgreen{orthogonal}, \forestgreen{constructor TRS} $\pair{\asig}{R}$, with
     %
  \item signature partition $\asig = \Ss \uplus \Sd$ into 
    \emph{stream symbols} and\\ \emph{data symbols}.
     %
\end{itemize}

\vspace*{2ex}
%
For the definition of stream spec's we also assume:
%
\begin{itemize}
  %  
  \item stream signature partition
    $\Ss = \{\sstrcns\} \uplus \Ssc \uplus \Ssf$,
    where 
    %
    \begin{itemize}
        %
        \vspace*{0.5ex}
        %
      \item $\text{`$\sstrcns$'}$ the \emph{stream constructor symbol},
        %with arity $(\sortD,\sortS) \to \sortS$,
        %
        \vspace*{0.5ex}
        %
      \item $\Ssc$ a set of \emph{stream constant symbols}
        having only data arguments;
        %
        \vspace*{0.5ex}
        %
      \item $\Ssf$ a set of \emph{stream function symbols} 
        with usually at least one stream argument.
        %
    \end{itemize}
    %
\end{itemize}

%% 
%% We assume, w.l.o.g., that for all $f \in \Ssf$
%% its stream arguments are in front;
%% that is,
%% $\arity{f} \in \sortS^{\arityS{f}}\sortD^{\arityD{f}} \to \{\sortS,\sortD\}$,
%% where $\arityS{f}$ and $\arityD{f} \in \nat$ 
%% are called the \emph{stream arity} and the \emph{data arity} of $f$, respectively.

\end{frame}%%%%%%%%%%%%%%%%%%%%%%%%%%%%%%%%%%%%%%%%%%%%%%%%%%%%%%%%%%%%%%


\begin{frame}%%%%%%%%%%%%%%%%%%%%%%%%%%%%%%%%%%%%%%%%%%%%%%%%%%%%%%%%%%%%
  \frametitle{Stream Specification}  


\vspace*{-0.65ex}
%
\begin{definition}
  %}
  Let $\atrs = \pair{\asig}{R}$ a stream TRS
  with part's $\asig = \Ssc \uplus \Ssf \uplus \{\sstrcns\} \uplus \Sd$
  and $R = \Rsc \uplus \Rsf \uplus \Rd$.
  $\atrs$ is a \magenta{stream specification} if:
  %
  \vspace*{-0.5ex}
  %
  \begin{center}
  \renewcommand{\arraystretch}{1.3}
  \begin{tabular}{crc}
    \cline{1-2}
    $\Rsc$ & {\em stream layer}\\
    \cline{1-2}
    $\Rsf$ & {\em function layer} & \multirow{2}{*}{$\Bigg\}$ \parbox{3.2cm}{\emph{stream function\\specification}}}\\
    \cline{1-2}
    $\Rd$ & {\em data layer}\\
    \cline{1-2}
  \end{tabular}
  \end{center}
  %
  \vspace*{-1.5ex}
  %
  \begin{itemize}
    \item\label{def:scs:1}
      The \emph{data-layer} $\atrs_{\mit{d}} = \pair{\Sd}{\Rd}$ is a terminating $\sortD$-sorted TRS.
    \item %Each layer may use symbols defined in a lower layer, but not vice-versa.
          The underlying \emph{stream function specification}\\
          $\atrs_{\mit{fun}} = \pair{\Ssf \uplus \{\sstrcns\} \uplus \Sd}{\Rsf \uplus \Rd}$
          is a TRS.
    \item 
      $\atrs$ is exhaustive for the defined symbols in $\asig$.
    \item
      $\Ssc$, a set of constant symbols, contains $\rootsc$, 
      the \emph{root of $\atrs$}.\\
      $\Rsc$ is the set of \emph{defining rules} $\rhom$: 
      $\astrcon\to\astrtrm$ for every $\astrcon\in\Ssc$.
      %
  \end{itemize}
  %
\end{definition}

\end{frame}%%%%%%%%%%%%%%%%%%%%%%%%%%%%%%%%%%%%%%%%%%%%%%%%%%%%%%%%%%%%%%


\begin{frame}%%%%%%%%%%%%%%%%%%%%%%%%%%%%%%%%%%%%%%%%%%%%%%%%%%%%%%%%%%%%
  \frametitle{Stream Specification (Layered setup)}
  
\begin{remark}
  %
  \begin{itemize}
    %
    \item every layer may use symbols from a lower layer,
      not vice versa
      %    
    \item \chocolate{isolated data-layer}:
      %
      \begin{itemize}
        %
        %\item prevents the use of stream symbols by data-rules
          %
        \item data-layer symbols are \emph{stream independent},
          excluding stream-dependent functions like:
          %
          \begin{equation*}
            %
            \strhd{\strcns{x}{\astr}} \red x 
            %
          \end{equation*}
          %
        \item by exhaustivity for $\Sd$ and strong normalization
          of $\atrs_{\mit{d}}$,\\
          closed data terms rewrite to constructor normal forms
          %
      \end{itemize}
      %
  \end{itemize}    
  %
\end{remark} 
      
\end{frame}%%%%%%%%%%%%%%%%%%%%%%%%%%%%%%%%%%%%%%%%%%%%%%%%%%%%%%%%%%%%%%

\begin{frame}%%%%%%%%%%%%%%%%%%%%%%%%%%%%%%%%%%%%%%%%%%%%%%%%%%%%%%%%%%%%
  \frametitle{Stream Specification (layered setup)}
  
\begin{remark}
  %
  \begin{itemize}
    %
    \item \chocolate{stream layer}: 
      %
      \begin{itemize}
        %
        \item only sortedness restrictions are imposed on how
          the rules in the stream layer make use of the symbols
          in the other layers. 
          %      
      \end{itemize} 
      %
    \item \chocolate{separate function layer}:
      %
      \begin{itemize}
        %
        \item we are interested in \forestgreen{managable} stream functions
           that define \emph{well-defined streams} or 
           finite \emph{stream prefixes}.
           
           The rule
           %
           \begin{align*}
             %
             \funap{\strff{f}}{\astr}
               & \red
             \strcns{\datf{0}}{\strcns{
               \strhd{\strtl{\funap{\strff{f}}{\astr}}}
                                       }{\funap{\strff{f}}{\astr}}}
               & & \text{(\alert{excluded!})}                        
               %
           \end{align*}
           %
           defines an operation on streams that produces
           an output stream with undefined odd elements.
           %
      \end{itemize}
      %
  \end{itemize}    
  %
\end{remark} 
      
\end{frame}%%%%%%%%%%%%%%%%%%%%%%%%%%%%%%%%%%%%%%%%%%%%%%%%%%%%%%%%%%%%%%


\begin{frame}%%%%%%%%%%%%%%%%%%%%%%%%%%%%%%%%%%%%%%%%%%%%%%%%%%%%%%%%%%%%
  \frametitle{Production of a term. Productivity of a stream spec.}


  Let $\conat \defdby \nat\cup\{\infty\}$ the
  \emph{coinductive natural numbers}. 

\begin{definition}
  %
  Let $\atrs = \pair{\asig}{R}$ a stream definition.  
  %
  \begin{itemize}
    %
    \item The \magenta{production} $\chocolate{\terprd{\atrs}{\darkcyan{t}}}$
      of a stream term $\darkcyan{t}  \in \ter{\asig}$: % is 
       %% the supremum of the number of data elements $\darkcyan{t}$ 
       %% can `produce':
      %
      \begin{equation*}
        %
        \chocolate{\terprd{\atrs}{\darkcyan{t} }}
          \defdby \supremum{\{\chocolate{n} \in \nat \mid  
          \darkcyan{t} \mred \lstcns{\adattrm_1}{\lstcns{\ldots}{\lstcns{\adattrm_{\chocolate{n}}}{t'}}}\}} 
          \in\conat \; .
           %
      \end{equation*}
      %
      \vspace*{-1ex}
      %
    \item $\atrs$ is called \magenta{productive} if
      $\terprd{\atrs}{\rootsc} = \infty$.
      %
  \end{itemize}
  %
\end{definition}\pause{}


\vspace*{1ex}
%
\begin{proposition}
  %
  A stream definition~$\atrs$ is productive if and only if%\\
  %\hspace*{\fill}
  %$ 
  \[
  \rootsc \infred 
           \lstcns{\adattrm_1}{\lstcns{\adattrm_2}
                              {\lstcns{\adattrm_3}
                              {\lstcns{\adattrm_4}{\ldots}}}}
  \]
  %$.
  %
\end{proposition}

\end{frame}%%%%%%%%%%%%%%%%%%%%%%%%%%%%%%%%%%%%%%%%%%%%%%%%%%%%%%%%%%%%%%


\begin{frame}%%%%%%%%%%%%%%%%%%%%%%%%%%%%%%%%%%%%%%%%%%%%%%%%%%%%%%%%%%%%
  \frametitle{Stream Specifications (Properties I)}

A stream spec $\atrs$ is called
%
\begin{itemize}
  %
  \item \magenta{flat}: 
    all rules in the function layer $\Rsf$ of $\atrs$ are \emph{flat}:
    no nested occurrences of stream function symbols on the right-hand side.
    \\[1ex]
    Excludes a rule: \hspace*{1ex}
     $ \funap{\strff{e}}{\strcns{x}{\astr}}
         \red
       \strcns{x}{\funap{\strff{e}}{\funap{\strff{e}}{\astr}}} $.
       % 
    \vspace*{2ex}
    %  
  \item \magenta{pure}:
    $\atrs$ is \emph{flat}, and for every symbol $\strff{f}\in\Ssf$
    the defining rules\\ of $\strff{f}$ in $\Rsf$ have the same
    consumption/production behaviour:\\
    they coincide (mod.\ renaming
    of variables) if all outermost data-subterms are replaced by $\pebble$.  
    \\[1ex]
    This excludes defining rules: \hspace*{1ex}
     \parbox[t]{115pt}{
     $ \strfunap{f}{ \strcns{\datf{0}}{\strcns{x}{\astr}} } 
         \red 
       \strcns{x}{\strcns{x}{\strfunap{f}{\strcns{\datf{0}}{\astr}}}} $\\
     $ \strfunap{f}{\strcns{\datf{1}}{\strcns{x}{\astr}}} 
         \red \strcns{x}{\strfunap{f}{\strcns{\datf{0}}{\astr}}} \; . $
       }
       % 
\end{itemize}

\end{frame}%%%%%%%%%%%%%%%%%%%%%%%%%%%%%%%%%%%%%%%%%%%%%%%%%%%%%%%%%%%%%%


\begin{frame}%%%%%%%%%%%%%%%%%%%%%%%%%%%%%%%%%%%%%%%%%%%%%%%%%%%%%%%%%%%%
  \frametitle{Stream Specification (flat, non-pure)}  


\begin{example}
  %
  % "a" -> "abc", "b" -> "ac", "c" -> "b"
  %
\begin{center}
  \renewcommand{\arraystretch}{1.3}
  %
  \scalebox{1}{
  %
  \begin{tabular}{r@{\hspace{1.5ex}}r@{\extracolsep{1ex}}c@{\extracolsep{1ex}}lr}
    \hline
    \multicolumn{4}{c}{
    $\strcf{Q} \red \strcns{\datf{a}}{\strcf{R}}$} & \multirow{2}{*}{\it{stream layer}}\\%[0.5ex]
    \multicolumn{4}{c}{
    $\strcf{R} \red \strcns{\datf{b}}{\strcns{\datf{c}}{\funap{\strff{f}}{\strcf{R}}}}$}
    \\
    \hline
    &$\funap{\strff{f}}{\strcns{\datf{a}}{\astr}}$ 
    &$\red$& $\strcns{\datf{a}}{\strcns{\datf{b}}{\strcns{\datf{c}}{\funap{\strff{f}}{\astr}}}}$
    \\%[0.5ex]
    &$\funap{\strff{f}}{\strcns{\datf{b}}{\astr}}$ 
    &$\red$& $\strcns{\datf{a}}{\strcns{\datf{c}}{\funap{\strff{f}}{\astr}}}$ & \it{function layer}\\%[0.5ex]
    &$\funap{\strff{f}}{\strcns{\datf{c}}{\astr}}$ 
    &$\red$& $\strcns{\datf{b}}{\funap{\strff{f}}{\astr}}$\\
    \hline
    & & & 
    & \it{data layer} \\
    \hline
  \end{tabular}
  %
  }
  %
\end{center}
  %
\end{example}

\vspace*{-1.5ex}
%
\begin{flushleft}
  \scalebox{1}{\ldots\ is \emph{productive} and specifies
               the \emph{ternary Thue-Morse sequence}:}\\[1.5ex]
  \hspace*{\fill}
  \scalebox{1}{
  $
  \strcf{Q}
  \infred
  \strcns{\datf{a}}{\strcns{\datf{b}}{\strcns{\datf{c}}{
  \strcns{\datf{a}}{\strcns{\datf{c}}{\strcns{\datf{b}}{
  \strcns{\datf{a}}{\strcns{\datf{b}}{\strcns{\datf{c}}{
  \strcns{\datf{b}}{\strcns{\datf{a}}{\strcns{\datf{c}}{
  \ldots}}}}}}}}}}}}
  $%
                  }
  \hspace*{\fill}\mbox{}
  %
\end{flushleft}

\end{frame}%%%%%%%%%%%%%%%%%%%%%%%%%%%%%%%%%%%%%%%%%%%%%%%%%%%%%%%%%%%%%%


\begin{frame}%%%%%%%%%%%%%%%%%%%%%%%%%%%%%%%%%%%%%%%%%%%%%%%%%%%%%%%%%%%%
  \frametitle{Stream Specification (flat, pure)}  

\begin{example}
%
%\vspace*{1.5ex}
%
\begin{center}
  %
\scalebox{1}{
  %
  \begin{tabular}{r@{\hspace{1.5ex}}r@{\extracolsep{1ex}}c@{\extracolsep{1ex}}lr}
    %
    \hline
    \multicolumn{4}{c}{\rule{0pt}{2ex}
    $\strcf{Q} \red \funap{\strff{diff}}{\strcf{M}}$}
    &\multirow{2}{*}{\it{stream layer} }
    \\[0ex]
    \multicolumn{4}{c}{
    $\strcf{M} \red \strcns{\datf{0}}{\strzip{\strinv{\strcf{M}}}{\strtl{\strcf{M}}}}$
    }
    \\
    \hline
    & \rule{0pt}{2ex} $\strzip{\strcns{x}{\astr}}{\bstr}$ 
    &$\red$& $\strcns{x}{\strzip{\bstr}{\astr}}$ \\[-0ex]
    & $\strinv{\strcns{x}{\astr}}$ 
    &$\red$& $\strcns{\inv{x}}{\strinv{\astr}}$
    &\multirow{2}{*}{\it{function layer}}
    \\[0ex]
    & $\strtl{\strcns{x}{\astr}}$ &$\red$& $\astr$ 
    \\[0ex]
    % 00 -> a , 01 -> b , 10 -> c, 11 -> a
    &$\strdiff{\strcns{x}{\strcns{y}{\astr}}}$ 
    &$\red$& $\strcns{\abc{x}{y}}{\strdiff{\strcns{y}{\astr}}}$
    \\[0ex]
    \hline
    & \rule{0pt}{2ex}
      $\inv{\datf{0}} \red \datf{1}$ & &$\inv{\datf{1}} \red \datf{0}$ \\[0ex]
    & $\abc{\datf{0}}{\datf{0}} \red \datf{b}$ & &$\abc{\datf{0}}{\datf{1}} \red \datf{a}$ 
    & \it{data layer}
    \\[0ex]
    & $\abc{\datf{1}}{\datf{0}} \red \datf{c}$ & &$\abc{\datf{1}}{\datf{1}} \red \datf{b}$ \\
    \hline
  \end{tabular}
  %
  }
  %
\end{center}
  %
  %\vspace*{0.75ex}
  %
\end{example}

\vspace*{-2.5ex}
%
\begin{flushleft}
  \scalebox{1}{\ldots\ is \emph{productive} and also specifies
               the \emph{ternary Thue-Morse sequence}.}
  %
\end{flushleft}

\end{frame}%%%%%%%%%%%%%%%%%%%%%%%%%%%%%%%%%%%%%%%%%%%%%%%%%%%%%%%%%%%%%%

\begin{frame}%%%%%%%%%%%%%%%%%%%%%%%%%%%%%%%%%%%%%%%%%%%%%%%%%%%%%%%%%%%%
  \frametitle{Stream Specification (flat, non-pure)}
  
\newcommand{\myvspace}{\rule{0pt}{2.2ex}}  
  %  
\begin{example}[Hamming numbers]% SFS for hamming
  %  
\begin{center}  
  %
\scalebox{0.8}{$ 
  \renewcommand{\arraystretch}{1.25} 
  %
\begin{array}{cr}
  \hline
    \myvspace%
     \strcf{H} 
    \to 
    \strcns{\trmrep{\datf{1}}}{
      \strmerge{
        \strtimes{\strcf{H}}{\trmrep{\datf{2}}}}{
        \strmerge{
          \strtimes{\strcf{H}}{\trmrep{\datf{3}}}}{
          \strtimes{\strcf{H}}{\trmrep{\datf{5}}}
        }
      }
    }
  & 
  \hspace{-2.5ex}
  \text{{\it stream layer}}
  \\[.5ex]
  \hline%
  %
  \begin{array}{rcr}
  \myvspace%  
  \strtimes{\strcns{x}{\astr}}{y} &\to& \strcns{\nummult{x}{y}}{\strtimes{\astr}{y}}
  \\
  \strmerge{\strcns{x}{\astr}}{\strcns{y}{\bstr}} &\to& \aux{\astr}{\bstr}{x}{y}{\compare{x}{y}}
  \\
  \aux{\astr}{\bstr}{x}{y}{\ltsymb} &\to& \strcns{x}{\strmerge{\astr}{\strcns{y}{\bstr}}}
  \\
  \aux{\astr}{\bstr}{x}{y}{\eqsymb} &\to& \strcns{x}{\strmerge{\astr}{\bstr}}
  \\
  \aux{\astr}{\bstr}{x}{y}{\gtsymb} &\to& \strcns{y}{\strmerge{\strcns{x}{\astr}}{\bstr}}
  \end{array}
  &
  \hspace{-2.5ex}
  \text{{\it function layer}}
  \\[6ex]
  \hline%
  %
  \begin{array}{rclcrcl}
  \myvspace%  
  \compare{\numzer}{\numzer} &\to& \eqsymb
  &\hspace{1em}&
  \numadd{\numzer}{y} &\to& y
  \\
  \compare{\numzer}{\numsuc{y}} &\to& \ltsymb
  &&
  \numadd{\numsuc{x}}{y} &\to& \numsuc{\numadd{x}{y}}
  \\
  \compare{\numsuc{x}}{\numzer} &\to& \gtsymb
  &&
  \nummult{\numzer}{y} &\to& \numzer
  \\
  \compare{\numsuc{x}}{\numsuc{y}} &\to& \compare{x}{y}
  &&
  \nummult{\numsuc{x}}{y} &\to& \numadd{y}{\nummult{x}{y}}
  \\
  \end{array}
  & 
  \hspace{-2.5ex}
  \text{{\it data layer}}
  \\[5ex]
  \hline%
\end{array}
  $
  }
  %
\end{center} 
  %
\end{example} 

\end{frame}%%%%%%%%%%%%%%%%%%%%%%%%%%%%%%%%%%%%%%%%%%%%%%%%%%%%%%%%%%%%%%


\begin{frame}%%%%%%%%%%%%%%%%%%%%%%%%%%%%%%%%%%%%%%%%%%%%%%%%%%%%%%%%%%%%
  \frametitle{Stream Specifications (Properties II)}

A stream spec $\atrs$ is called
%
\begin{itemize}
  %
  \item \magenta{friendly-nesting}:
    all rules in the function layer $\Rsf$ are \emph{flat},
    or contained in a subset $\tilde{R}\subset\Rsf$ of 
    \emph{friendly-nesting} rules:\\
    every $\rho\in\tilde{R}$
    %
    \begin{itemize}
      %
      \item consumes in each stream argument at most one element,
      \item produces at least one, and
      \item all defining rules of stream functions occurring on the
        right-hand side of $\rho$ are again in $\tilde{R}$.
        %
    \end{itemize}
    %
\end{itemize}



%
\begin{example}
%
\begin{center}
  $
  \bfunap{\strff{f}}{\strcns{x}{\astr}}{\bstr}
     \red 
      \strcns{x}{\strcns{x}{\funap{\strff{g}}{\bfunap{\strff{f}}{\astr}{\strcns{x}{\bstr}}}}}
  $
  \\[0.5ex]
  $
  \funap{\strff{g}}{\strcns{x}{\astr}}
     \red
      \strcns{x}{\funap{\strff{g}}{\strcns{x}{\bfunap{\strff{f}}{\astr}{\astr}}}}
  $
\end{center}
%
\end{example}

\end{frame}%%%%%%%%%%%%%%%%%%%%%%%%%%%%%%%%%%%%%%%%%%%%%%%%%%%%%%%%%%%%%%


\begin{frame}%%%%%%%%%%%%%%%%%%%%%%%%%%%%%%%%%%%%%%%%%%%%%%%%%%%%%%%%%%%%
  \frametitle{Stream Specification (friendly-nesting)}
  
\newcommand{\myvspace}{\rule{0pt}{2.2ex}}  
\renewcommand{\arraystretch}{1.25}
  %  
  
\begin{example}
  %
  \vspace*{-0.5ex}
  %
\begin{center}  
  %
\scalebox{.93}{$  
  %
  \begin{array}{cr} 
    %
    \hline
    %
    \begin{array}{c}
      %
      \nats \red \strcns{\datf{0}}{\convprod{\ones}{\ones}}
      \\
      \ones \red \strcns{\numsuc{\datf{0}}}{\ones}
      %
    \end{array}
    %
    & \text{{\it stream layer}}
    \\
    %
    \hline%
    %
    \begin{array}{c}
      %
      \bfunap{\strff{\boldsymbol{\times}}}%
        {\strcns{x}{\astr}}{\strcns{y}{\bstr}}
       \red   \strcns{\nummult{x}{y}}%
                       {\stradd{ \strtimes{\bstr}{x} }%
                               { \bfunap{\strff{\boldsymbol{\times}}}%
                                        {\astr}{\strcns{y}{\bstr}} } }
                                                                       \\
      \strtimes{\strcns{x}{\astr}}{y}%
        \red  \strcns{\nummult{x}{y}}{\strtimes{\astr}{y}}
                                                                       \\
      \stradd{\strcns{x}{\astr}}{\strcns{y}{\bstr}}%
        \red   \strcns{\numadd{x}{y}}%
                        {\stradd{\astr}{\bstr}}
      %
    \end{array}
    %
    & \text{{\it function layer}}
    \\%[3.5ex]
    %
    \hline%
    %
    \begin{array}{cc}
      %
      \numadd{x}{\datf{0}} \red   x    
      & 
      \numadd{x}{\numsuc{y}}   \red   \numsuc{\numadd{x}{y}} 
      \\
      \nummult{x}{\datf{0}} \red   \datf{0}  
      & 
      \nummult{x}{\numsuc{y}}   \red   
          \numadd{ \nummult{x}{y} }{x}
      %
    \end{array}
    %
    & \text{{\it data layer}} 
    \\
    \hline
    %
  \end{array}
  %
  $}
  %
  \vspace*{-1.5ex}
  %
\end{center}
  %
\end{example}

$\sconvprod$ defines the \emph{convolution product} stream operation
$ \pair{\astr}{\bstr} \mapsto \astr\times\bstr $:
%
\begin{align*}
  %
  \strnth{({\astr}\times{\bstr})}{i} 
    & = \sum_{j=0}^i \strnth{\astr}{j}\cdot\strnth{\bstr}{i-j} 
  & & \text{(for all $i\in\nat$)}
  %
\end{align*}


\end{frame}%%%%%%%%%%%%%%%%%%%%%%%%%%%%%%%%%%%%%%%%%%%%%%%%%%%%%%%%%%%%%%



\begin{frame}%%%%%%%%%%%%%%%%%%%%%%%%%%%%%%%%%%%%%%%%%%%%%%%%%%%%%%%%%%%%
  \frametitle{Map of Stream Specifications}  

    
\begin{center}
  %
  \scalebox{1.15}{ 
    \includegraphics{figs/pstricks/pnp}
              }
  %
\end{center}

\end{frame}%%%%%%%%%%%%%%%%%%%%%%%%%%%%%%%%%%%%%%%%%%%%%%%%%%%%%%%%%%%%%%





%------------------------------------------------------------------------
%------------------------------------------------------------------------
\section{Data-Oblivious Analysis}
%------------------------------------------------------------------------
%------------------------------------------------------------------------


%------------------------------------------------------------------------
\subsection{Intuition}
%------------------------------------------------------------------------


\begin{frame}%%%%%%%%%%%%%%%%%%%%%%%%%%%%%%%%%%%%%%%%%%%%%%%%%%%%%%%%%%%%
  \frametitle{Data-Oblivious Analysis}

\begin{center}
  %
\scalebox{1}{
\renewcommand{\arraystretch}{1.25}
\begin{tabular}{r@{\hspace{1.5ex}}cr}
  \hline
  &
  $\strcf{T} \red \funap{\strff{f}}{\strcns{\datf{0}}{\strcns{\datf{1}}{\strcf{T}}}}$
  & {\it stream layer}
  \\
  \hline
  $(\rho_{\astrfunsub\datf{0}}):$\hspace*{2ex}
  &
  $\strfunap{f}{ \strcns{\datf{0}}{\strcns{x}{\astr}} } 
   \red \strcns{\datf{0}}{\strcns{\datf{1}}{\strfunap{f}{\astr}}}$
  \\
  $(\rho_{\astrfunsub\datf{1}}):$\hspace*{2ex} 
  &
  $\strfunap{f}{\strcns{\datf{1}}{\strcns{x}{\astr}}} 
   \red \strcns{x}{\strfunap{f}{\astr}}$
  & {\it function layer}
  \\
  \hline
  & & {\it data layer}\\
  \hline
\end{tabular}
  %
  }
  %
\end{center}


This specification is \emph{productive}:
%
\begin{equation*}
  %
  \strcf{T} 
    \mred \strcns{\datf{0}}{\strcns{\datf{1}}{\funap{\strff{f}}{\strcf{T}}}}
    \mred \strcns{\datf{0}}{\strcns{\datf{1}}{\strcns{\datf{0}}{\strcns{\datf{1}}{\funap{\strff{f}}{\funap{\strff{f}}{\strcf{T}}}}}}}
    \mred \ldots 
    \infred \strcns{\datf{0}}{\strcns{\datf{1}}{\strcns{\datf{0}}{\strcns{\datf{1}}{\ldots}}}} \; ,
    %
\end{equation*}
%
but, disregarding the identity of data, the rewrite sequence:
%
\begin{equation*}
  %
  \strcf{T}
    \red \funap{\strff{f}}{\strcns{\trspeb}{\strcns{\trspeb}{\strcf{T}}}}
    \red^{\rho_{\astrfunsub\datf{1}}}
    \strcns{\trspeb}{\funap{\strff{f}}{\strcf{T}}}
    \mred 
    \ldots
    \infred \strcns{\trspeb}{\funap{\strff{f}}{\strcns{\trspeb}{\funap{\strff{f}}{\strcns{\trspeb}{\funap{\strff{f}}{\ldots}}}}}} \; .
    %
\end{equation*}
%
is possible.
Hence the specification is 
\alert{not} \emph{data-obliviously productive}.

\end{frame}%%%%%%%%%%%%%%%%%%%%%%%%%%%%%%%%%%%%%%%%%%%%%%%%%%%%%%%%%%%%%%


\begin{frame}%%%%%%%%%%%%%%%%%%%%%%%%%%%%%%%%%%%%%%%%%%%%%%%%%%%%%%%%%%%%

What is a data-oblivious analysis of productivity?
\begin{itemize}
  \item `quantitative reasoning'
  \item knowledge about concrete values of data elements is ignored
  \item abstract from the concrete data values
\end{itemize}

\begin{example}
\vspace{-3ex}
\begin{align*}
  %
  \strfunap{f}{ \strcns{\datf{0}}{\strcns{x}{\astr}} } 
  &\red \strcns{x}{\strcns{x}{\strfunap{f}{\strcns{\datf{0}}{\astr}}}} 
  & 
  \strfunap{f}{ \strcns{\pebble}{\strcns{\pebble}{\astr}} } 
  &\red \strcns{\pebble}{\strcns{\pebble}{\strfunap{f}{\strcns{\pebble}{\astr}}}}  \\
  \strfunap{f}{\strcns{\datf{1}}{\strcns{x}{\astr}}} 
  &\red \strcns{x}{\strfunap{f}{\strcns{\datf{0}}{\astr}}}
  &
  \strfunap{f}{\strcns{\pebble}{\strcns{\pebble}{\astr}}} 
  &\red \strcns{\pebble}{\strfunap{f}{\strcns{\pebble}{\astr}}}
  %
\end{align*}
Taking into account that $\datf{0}$ is supplied to the recursive call:
\begin{itemize}
  \item $n \mapsto \cosubtr{2n}{3}$ is the tight lower bound on the production relation
\end{itemize}
However, a data-oblivious analysis ignores this information:
\begin{itemize}
  \item $n \mapsto \cosubtr{n}{1}$ is the data-oblivious lower bound
\end{itemize}
\end{example}

\end{frame}%%%%%%%%%%%%%%%%%%%%%%%%%%%%%%%%%%%%%%%%%%%%%%%%%%%%%%%%%%%%%%


\begin{frame}%%%%%%%%%%%%%%%%%%%%%%%%%%%%%%%%%%%%%%%%%%%%%%%%%%%%%%%%%%%%

We formalise data-oblivious term rewriting as \emph{two-player game}:
\begin{itemize}
  \item \emph{rewrite player} $\mathcal{R}$ performs the usual term rewriting
  \item \emph{data-exchange player} $\sdg$ arbitrarily exchanges data elements
\end{itemize}

\begin{example}[
  $\strfunap{f}{ \strcns{\datf{0}}{\strcns{x}{\astr}} }
   \red \strcns{x}{\strcns{x}{\strfunap{f}{\strcns{\datf{0}}{\astr}}}},\;
   \strfunap{f}{\strcns{\datf{1}}{\strcns{x}{\astr}}} 
   \red \strcns{x}{\strfunap{f}{\strcns{\datf{0}}{\astr}}}$
]
Data-oblivious rewriting of the term \emph{$\strfunap{f}{\strcns{\datf{0}}{\strcns{\datf{1}}{\strcns{\datf{0}}{\astr}}}}$}:
\vspace{-1ex}
\begin{center}
\scalebox{.7}{\includegraphics{figs/pstricks/do_example1}}
\end{center}
\vspace{-1ex}
\end{example}
\begin{definition}[data-oblivious lower (upper) bound on the production]
\ldots of a term $s$ is the infimum (supremum) of the production of $s$\\
with respect to all possible strategies for the data-exchange player $\sdg$.
\end{definition}

\end{frame}%%%%%%%%%%%%%%%%%%%%%%%%%%%%%%%%%%%%%%%%%%%%%%%%%%%%%%%%%%%%%%


\begin{frame}%%%%%%%%%%%%%%%%%%%%%%%%%%%%%%%%%%%%%%%%%%%%%%%%%%%%%%%%%%%%

It is sufficient to use `data-value-invariant` strategies for $\sdg$:
\begin{itemize}
  \item for $\sdg$ we can abstract from the data elements in favour of $\trspeb$
\end{itemize}
\begin{definition}[The data abstraction $\databstr{s}$ of a term $s$]
   \ldots{} is obtained from $s$ by replacing all data terms in $s$ with $\trspeb$.
\end{definition}
\begin{example}
\vspace{-1.5ex}
\begin{center}
\scalebox{.7}{\includegraphics{figs/pstricks/do_example1a}}
\end{center}
\vspace{-1.5ex}
\end{example}
\begin{definition}[The data-guess function $\sdg$]
  \ldots instantiates all $\trspeb$ by closed data terms in constructor normal form.
\end{definition}

\end{frame}%%%%%%%%%%%%%%%%%%%%%%%%%%%%%%%%%%%%%%%%%%%%%%%%%%%%%%%%%%%%%%


%------------------------------------------------------------------------
\subsection{Definition and Properties}
%------------------------------------------------------------------------

\begin{frame}%%%%%%%%%%%%%%%%%%%%%%%%%%%%%%%%%%%%%%%%%%%%%%%%%%%%%%%%%%%%

\begin{definition}
  The \emph{data-oblivious production range} of a term $s \in \ter{\asig}_{\sortS}$ is:
  \begin{align*}
  \doRng{\atrs}{\astrtrm} \defdby  \{ \terprd{\sdg}{\databstr{\astrtrm}} \mid{}
     &\sdg \text{ a data-guess function on } \atrs \} \;.
  \end{align*}
  The \emph{data-oblivious lower} and \emph{upper bound on the production of $s$:}
  \begin{align*}
  \doLow{\atrs}{\astrtrm} &\defdby \inf(\doRng{\atrs}{\astrtrm})\\
  \doUp{\atrs}{\astrtrm} &\defdby \sup(\doRng{\atrs}{\astrtrm})
  \end{align*}

\end{definition}%%%%%%%%%%%%%%%%%%%%%%%%%%%%%%%%%%%%%%%%%%%%%%%%%%%%%%%%%


\begin{definition}%%%%%%%%%%%%%%%%%%%%%%%%%%%%%%%%%%%%%%%%%%%%%%%%%%%%%%%

  An stream specification~$\atrs$ is called
  \begin{itemize}
    \item \emph{data-obliviously productive} if $\doLow{\atrs}{\rootsc} = \infty$
    \item \emph{data-obliviously non-productive} if $\doUp{\atrs}{\rootsc} < \infty$
  \end{itemize}
\end{definition}

\end{frame}%%%%%%%%%%%%%%%%%%%%%%%%%%%%%%%%%%%%%%%%%%%%%%%%%%%%%%%%%%%%%%


\begin{frame}%%%%%%%%%%%%%%%%%%%%%%%%%%%%%%%%%%%%%%%%%%%%%%%%%%%%%%%%%%%%
  \frametitle{Data-Oblivious Productivity versus Productivity}
  
\begin{proposition}
  %
  Let\/ $\atrs = \pair{\asig}{R}$ be a stream specification.
  %
  \begin{itemize}
    %
    \item For all stream terms $s \in \ter{\asig}_{\sortS}$,
      we have
      %
      \begin{equation*}
        %
        \doLow{\atrs}{s} \le \terprd{\atrs}{s} \le \doUp{\atrs}{s} \; .
        %
      \end{equation*}
      %
  \end{itemize}
  %
  Hence:
  %
  \begin{itemize}
    %
    \item data-oblivious productivity implies productivity;
      %
    \item data-oblivious non-productivity implies non-productivity.
    %
  \end{itemize}
  %
\end{proposition}

\end{frame}%%%%%%%%%%%%%%%%%%%%%%%%%%%%%%%%%%%%%%%%%%%%%%%%%%%%%%%%%%%%%%  



%------------------------------------------------------------------------
%------------------------------------------------------------------------
\section[Production Calculus]{The Production Calculus}
%------------------------------------------------------------------------
%------------------------------------------------------------------------


\begin{frame}%%%%%%%%%%%%%%%%%%%%%%%%%%%%%%%%%%%%%%%%%%%%%%%%%%%%%%%%%%%%
  \frametitle{Periodically Increasing Functions}

\begin{definition}
  %
  Let $f  \funin \nat \to \conat$.

  $f$ is \emph{eventually periodic} if 
  $ \langle f(0), f(1), f(2), ...\rangle$ is eventually periodic.

  $f$ is \emph{periodically increasing} if it is non-decreasing,
  and its derivative $f' : \nat\to\conat$  with $f'(n) = f(n+1) - f(n)$
  is eventually periodic.
  %
\end{definition}

\begin{itemize}
  %
  \item \emph{Representation}:  
    pairs $\pair{\alpha}{\beta}\in\emph{\boldsymbol{\sqpair}}$
    with $\alpha\in\{\ioin,\ioout\}^*$, $\beta\in\{\ioin,\ioout\}^+$
    where $\ioout$ stands for output, $\ioin$ for input.
    %
  \item \emph{production function} $\siosqprd{\pair{\alpha}{\beta}}$
    for $\pair{\alpha}{\beta}\in\sqpair$:
    if $ s = \alpha\beta\beta\beta\ldots $ % $ \in\{\ioin,ioout\}^{\omega} $, 
    then
    \[
      \iosqprd{\pair{\alpha}{\beta}}{n} \defdby
        \begin{cases}
          %
          \text{number of ``$\ioout$'' from left 
               until the $(n+1)$-th ``$\ioin$'' in $l$}
            \\
          \conattop \hspace*{2ex} 
                    \text{\ldots\ if there are less than $n+1$ symbols in $l$}
            %
        \end{cases}
    \]
    %
  \item Abbreviation: $\alpha\iosqcyc{\beta}$ for $\pair{\alpha}{\beta}$.\\
    Example: identity function is represented by $\iosqcyc{\ioin\ioout}$. 
    %
\end{itemize}

\end{frame}%%%%%%%%%%%%%%%%%%%%%%%%%%%%%%%%%%%%%%%%%%%%%%%%%%%%%%%%%%%%%%  


\begin{frame}%%%%%%%%%%%%%%%%%%%%%%%%%%%%%%%%%%%%%%%%%%%%%%%%%%%%%%%%%%%%
  \frametitle{Production Terms}

\vspace*{-0.45ex}
%
\begin{definition}
  %
  For $\nam$ a set of recursion variables,
  the set $\net$ of \emph{production terms} is generated by:
  \[
    \anet
    \BNFis
    \netsrc{k}
    \BNFor
    \netvar{\avar}
    \BNFor
    \netbox{\aiosq}{\anet}
    \BNFor
    \netrec{\anam}{\anet}
    \BNFor
    \netmeet{\anet}{\anet}
  \]
  where $\anam\in\nam$, $\aiosq\in\sqpair$,
  and $\netsrc{k}$ is a \emph{numeral} for $k\in\conat$.
  %
  %% For $\anet_1,\ldots,\anet_n\in\net$ and $n\in\nat$, 
  %% we use $\netmeetn{n}{\anet_1,\ldots,\anet_n}$
  %% as shorthand for the production term
  %% $\netmeet{\anet_1}{\netmeet{\anet_2}{\ldots,\netmeet{\anet_{n-1}}{\anet_n}}}$.

  \vspace*{0.5ex}
  %
  The \emph{production} $\chocolate{\netbhv{\black{\anet}}} \in \conat$
  of a closed production term $\anet \in \net$ is defined by 
  induction on the term structure,
  interpreting:
  \begin{itemize}
    %
    \item $\snetrec$ as the least fixed point operator,
    \item $\aiosq$ as $\siosqprd{\aiosq}$, 
    \item $\netsrc{k}$ as $k$, and 
    \item $\snetmeet$ as $\min$.
    %
  \end{itemize}
  %
  \emph{$r$-ary Gates}:  production term contexts 
    $\netmeetn{r}{\netbox{\iaiosq{1}}{\porti{1}},\ldots,\netbox{\iaiosq{r}}{\porti{r}}}$.
  %
\end{definition}

\end{frame}%%%%%%%%%%%%%%%%%%%%%%%%%%%%%%%%%%%%%%%%%%%%%%%%%%%%%%%%%%%%%%  


\begin{frame}%%%%%%%%%%%%%%%%%%%%%%%%%%%%%%%%%%%%%%%%%%%%%%%%%%%%%%%%%%%%
  \frametitle{Reduction $\scolred$}

\begin{definition}
  The \emph{reduction relation 
  $\scolred$ on production terms}
  is defined as the compatible closure of: 
  %
  \begin{align*}
    %
    \netbox{\iaiosq{1}}{\netbox{\iaiosq{2}}{\anet}}
    & \red \netbox{\iosqcmp{\iaiosq{1}}{\iaiosq{2}}}{\anet}
    \pagebreak[0]\\
    \netbox{\aiosq}{\netmeet{\ianet{1}}{\ianet{2}}}
    & \red \netmeet{\netbox{\aiosq}{\ianet{1}}}{\netbox{\aiosq}{\ianet{2}}}
    \pagebreak[0]\\
    \netrec{\anam}{\netmeet{\ianet{1}}{\ianet{2}}}
    & \red \netmeet{\netrec{\anam}{\ianet{1}}}{\netrec{\anam}{\ianet{2}}}
    \pagebreak[0]\\
    \netrec{\anam}{\anet}
    & \red \anet
    \qquad \text{if $\anam\not\in\netfv{\anet}$}
    \pagebreak[0]\\
    \netrec{\anam}{\netbox{\aiosq}{\anam}}
    & \red \netsrc{\iosqfix{\aiosq}}
    \pagebreak[0]\\
    \netrec{\anam}{\netvar{\anam}}
    & \red \netsrc{0}
    \pagebreak[0]\\
    \netbox{\aiosq}{\netsrc{k}}
    &\red
    \netsrc{\iosqprd{\aiosq}{k}}
    \pagebreak[0]\\
    \netmeet{\netsrc{k_1}}{\netsrc{k_2}}
    & \red \netsrc{\comin{k_1,k_2}}  
    %
  \end{align*}
  %
\end{definition}

\end{frame}%%%%%%%%%%%%%%%%%%%%%%%%%%%%%%%%%%%%%%%%%%%%%%%%%%%%%%%%%%%%%%  


\begin{frame}%%%%%%%%%%%%%%%%%%%%%%%%%%%%%%%%%%%%%%%%%%%%%%%%%%%%%%%%%%%%
  \frametitle{Reduction $\scolred$}

Properties of $\scolred$:
%
\begin{itemize}
  %
  \item production preserving;
    %
  \item confluent and terminating;
    %
  \item normal forms are numerals.
  %
\end{itemize}

\vspace*{2ex}
%
\begin{theorem}
  %
  For all $\anet\in\net$:
  \[ \netprd{\anet}=k \; ,\]
  where $\netsrc{k}$ is the uniquely determined
  $\scolred$-normal form of $\anet$.
  %
\end{theorem}


\end{frame}%%%%%%%%%%%%%%%%%%%%%%%%%%%%%%%%%%%%%%%%%%%%%%%%%%%%%%%%%%%%%%  


%------------------------------------------------------------------------
%------------------------------------------------------------------------
\section[Translation]{Translation into Production Terms}
%------------------------------------------------------------------------
%------------------------------------------------------------------------


\begin{frame}%%%%%%%%%%%%%%%%%%%%%%%%%%%%%%%%%%%%%%%%%%%%%%%%%%%%%%%%%%%%
  \frametitle{Translation into Production Terms}


%\begin{itemize}
  %
  %\item
    We use a translation 
    that maps every \forestgreen{flat} stream spec $\atrs$
    with root $\rootsc$ to a \emph{production term} $\trnsl{\rootsc}$
    such that
    %
    \begin{equation*}
      %
      \doLow{\atrs}{\rootsc} = \netbhv{\trnsl{\rootsc}} \; .
      %
    \end{equation*}
    %
%\end{itemize}


\begin{itemize}
  %
  \item \emph{function layer translation}:
    Obtain a family $\{\trnsl{\astrfun} \}_{\astrfun\in\Ssf}$
    of gates such that, for every $\strff{f}\in\Ssf$, the gate $\fgate$
    represents the data-oblivious lower bound of $\strff{f}$ in $\atrs$.

    \vspace*{0.5ex}
    %
    (Involves solving an originally infinite `\ioterm\ specification'.)
    %
    % (Involves solving, for each argument place in $\strff{f}$
    %  an originally infinite `\ioterm\ specification' which
    %  specifies the data-oblivious lower bound in that argument place.)
    %
    \vspace*{1ex}
    %
  \item \emph{stream layer translation}:
    Using the family $\strff{f}\in\Ssf$ of gates, obtain 
    a production term $\trnslF{\rootsc}{\afam}$ such that
    $ \netbhv{\trnslF{\rootsc}{\afam}} = \doLow{\atrs}{\rootsc} $. 

    \vspace*{0.5ex}
    %
    (Involves expanding the stream layer rules step by step and
     a finite loop-checking procedure.)
    %
\end{itemize}

\end{frame}%%%%%%%%%%%%%%%%%%%%%%%%%%%%%%%%%%%%%%%%%%%%%%%%%%%%%%%%%%%%%%  


%------------------------------------------------------------------------
%------------------------------------------------------------------------
\section[Deciding D-O Productivity]{Deciding Data-Oblivious Productivity}
%------------------------------------------------------------------------
%------------------------------------------------------------------------



\begin{frame}%%%%%%%%%%%%%%%%%%%%%%%%%%%%%%%%%%%%%%%%%%%%%%%%%%%%%%%%%%%%
  \frametitle{Algorithm \algdecidedoprod:
              Deciding Data-Oblivious Productivity}
  
  \begin{enumerate}
    %
    \item Take as input:
      a \emph{flat} stream specification~$\atrs = \pair{\asig}{R}$.
      %
      \vspace*{1.25ex}
      %
    \item  Compute the translation of stream function symbols $\astrfun$\\ 
      into gates $\trnsl{\astrfun}$, yielding a family 
      $\afam \defdby \{\trnsl{\astrfun} \}_{\astrfun\in\Ssf}$ of gates. 
      %
      \vspace*{1.25ex}
      %
    \item  Construct the production term $\trnslF{\rootsc}{\afam}$ of the root $\rootsc$ of $\atrs$
    with respect to the family of gates $\afam$.
      %
      \vspace*{1.25ex}
      %
    \item Compute the production $\netsrc{k}$ of $\trnslF{\rootsc}{\afam}$ 
      using the reduction rel.~$\scolred$.
      %
      \vspace*{1.25ex}
      %
    \item  Give the following output:\\
      \hspace*{1ex}
      If $k=\conattop$: 
        ``$\atrs$ \texttt{is data-obliviously productive}'';\\
      \hspace*{1ex}
      else 
        $k\in\nat$: ``$\atrs$ \texttt{is not data-obliviously productive}''.
      %
  \end{enumerate}

\end{frame}%%%%%%%%%%%%%%%%%%%%%%%%%%%%%%%%%%%%%%%%%%%%%%%%%%%%%%%%%%%%%%  




\begin{frame}%%%%%%%%%%%%%%%%%%%%%%%%%%%%%%%%%%%%%%%%%%%%%%%%%%%%%%%%%%%%
  \frametitle{Deciding D-O Productivity. Recognising Productivity.}  


\begin{theorem} 
  %
  Data-oblivious productivity of \emph{flat} stream specifications 
  is decidable.

  I.p.,  the algorithm\/~$\algdecidedoprodem$ 
  decides data-oblivious productivity of flat stream specifications. 
  %
\end{theorem}

\vspace*{0.75ex}
%
Since data-oblivious productivity implies productivity,
we get a 
\chocolate{computable, data-obliviously optimal criterion for productivity}:
%
\vspace*{0.75ex}
%
\begin{corollary}
  %
  A \emph{flat} stream specification $\atrs$ \chocolate{is productive} 
  if the algorithm $\algdecidedoprodem$ recognizes $\atrs$
  as data-obliviously productive.
  %
\end{corollary}


\end{frame}%%%%%%%%%%%%%%%%%%%%%%%%%%%%%%%%%%%%%%%%%%%%%%%%%%%%%%%%%%%%%%


\begin{frame}%%%%%%%%%%%%%%%%%%%%%%%%%%%%%%%%%%%%%%%%%%%%%%%%%%%%%%%%%%%%
  \frametitle{Recognising and Deciding Productivity.}  


For \emph{pure} stream spec's productivity coincides with 
data-oblivious productivity. Hence $\algdecidedoprod$ gives rise
to a \chocolate{decision algorithm}. 

\begin{theorem}
  %
  Productivity is decidable for \emph{pure} stream spec's. 
  %
\end{theorem}

\vspace*{2.5ex}
%
Furthermore, a variant of $\algdecidedoprod$ can be used 
as a:
\begin{itemize}
  %
  \item \chocolate{computable criterion of productivity} for
    \emph{friendly-nesting} stream~spec's.
    %
\end{itemize}

\end{frame}%%%%%%%%%%%%%%%%%%%%%%%%%%%%%%%%%%%%%%%%%%%%%%%%%%%%%%%%%%%%%%


\begin{frame}%%%%%%%%%%%%%%%%%%%%%%%%%%%%%%%%%%%%%%%%%%%%%%%%%%%%%%%%%%%%
  \frametitle{Map of Stream Specifications}  

\begin{center}
  %
  \scalebox{1.15}{ 
    \includegraphics{figs/pstricks/pnp}
              }
  %
\end{center}

\end{frame}%%%%%%%%%%%%%%%%%%%%%%%%%%%%%%%%%%%%%%%%%%%%%%%%%%%%%%%%%%%%%%




%------------------------------------------------------------------------
%------------------------------------------------------------------------
\section[Conclusion]{Conclusion}
%------------------------------------------------------------------------
%------------------------------------------------------------------------


%------------------------------------------------------------------------
\subsection{Summary. Further Plans.}
%------------------------------------------------------------------------


\begin{frame}%%%%%%%%%%%%%%%%%%%%%%%%%%%%%%%%%%%%%%%%%%%%%%%%%%%%%%%%%%%%
  \frametitle{Summary}


\begin{itemize}%[<+->]
    %
  \item
  \chocolate{Previous Approaches}: 
  \emph{sufficient conditions} for productivity,\\
  not automatable or only for a limited subclass
    %
  \item
  \chocolate{FCT'07-Paper}:
  \emph{decision algorithm} for productivity of \darkmagenta{pure} stream 
     spec's
     %
  \item
  \chocolate{New Results}:
    %
    \begin{enumerate}
      %
      \vspace*{0.75ex}
        %
      \item\label{item:summary:flat}
        a \emph{computable, data-obliviously optimal,
        sufficient condition} for productivity
        of \darkmagenta{flat} stream spec's;
        %
        \vspace*{0.5ex}
        %
      \item\label{item:summary:pure}
        a \emph{decision method} for productivity on 
        \darkmagenta{pure} stream spec's
        with \darkmagenta{duplication/additional supply}
        in stream arg's (extension of FCT);
        %
        \vspace*{0.5ex}
        %
      \item\label{item:summary:friendly} 
        an extension of~\ref{item:summary:flat}
        to stream spec's with \darkmagenta{friendly nesting},
        disregarding data-oblivious optimality;
        %
        \vspace*{0.5ex}
        %
      \item a tool automating~\ref{item:summary:flat}, \ref{item:summary:pure} 
        and~\ref{item:summary:friendly} available at:
        \href{http://localhost/productivity/}%
         {\emph{\tt http://infinity.few.vu.nl/productivity}}.
         %
   \end{enumerate}
   %
\end{itemize}

\end{frame}%%%%%%%%%%%%%%%%%%%%%%%%%%%%%%%%%%%%%%%%%%%%%%%%%%%%%%%%%%%%%%  


\begin{frame}%%%%%%%%%%%%%%%%%%%%%%%%%%%%%%%%%%%%%%%%%%%%%%%%%%%%%%%%%%%%
  \frametitle{Present and Future Work}

  \begin{itemize}
    %
    \item A precise \emph{complexity analysis} of our algorithms.
      %
      \vspace*{0.75ex}
      %
    \item \emph{Data-aware methods} for recognising productivity.
      %
      \vspace*{0.75ex}
      %
    \item A \emph{refined pebbleflow semantics} that accounts for the
      delay of evaluation of stream elements as made possible
      by lazy evaluation strategies.
      Think of Sijtsma's example:
      $
  \strcf{S} \to \strcns{\datf{0}}{\strcns{\strhd{\strtln{2}{\strcf{S}}}}{\strcf{S}}}
      $.
      \vspace*{0.75ex}
      %
    \item A theory of \emph{reducibility between streams}.
      %
      \vspace*{0.75ex}
      %
    \item Can our results be used to obtained general results clarifying
      under which conditions \emph{term graph rewriting} can be viewed
      as a \emph{semantics for infinitary rewriting}?
      %
  \end{itemize}

\end{frame}%%%%%%%%%%%%%%%%%%%%%%%%%%%%%%%%%%%%%%%%%%%%%%%%%%%%%%%%%%%%%%  



%------------------------------------------------------------------------
\subsection{Tools}
%------------------------------------------------------------------------


\begin{frame}%%%%%%%%%%%%%%%%%%%%%%%%%%%%%%%%%%%%%%%%%%%%%%%%%%%%%%%%%%%%
  \frametitle{Our Papers and Tools.}

  \vspace*{0.5ex}
  %
  Please visit \href{http://localhost/productivity/}%
         {\emph{\tt http://infinity.few.vu.nl/productivity}}\\
  to find: 
  %
  \vspace*{1.5ex}
  %
  \begin{itemize}
    %
    \item 
      Endrullis, Grabmayer, Hendriks, Isihara, Klop:\\
      \emph{Productivity of Stream Definitions},
      Proceedings of FCT~2007, LNCS~4637, pages~274--287, 2007;
      %
      \vspace*{0.75ex}
      %
    \item  
      Endrullis, Grabmayer, Hendriks, Isihara, Klop:\\
      \emph{Productivity of Stream Definitions},
      journal submission;
      %
      \vspace*{0.75ex}
      %
    \item Endrullis, Grabmayer, Hendriks:\\
      \emph{Data-Oblivious Stream Productivity},
      extended abstract;
      %
      \vspace*{0.75ex}
      %
    \item access to our tools:
      %
      \begin{itemize}
        %
        \item Endrullis: tool implementing the\\ 
          \hspace*{3ex}  
          \chocolate{decision algorithm
                     for data-oblivious productivity};
          %
        \item Isihara: \chocolate{pebbleflow visualization} tool.

        %
      \end{itemize} 
      %
  \end{itemize}

\end{frame}%%%%%%%%%%%%%%%%%%%%%%%%%%%%%%%%%%%%%%%%%%%%%%%%%%%%%%%%%%%%%


\begin{frame}%%%%%%%%%%%%%%%%%%%%%%%%%%%%%%%%%%%%%%%%%%%%%%%%%%%%%%%%%%%%
  \frametitle{}

   \vspace*{\fill}
   %
   \hspace{\fill}%
   \scalebox{1.6}{\forestgreen{\bf Thanks for your attention!}}%
   \hspace*{\fill}\mbox{}
   %
   \label{lastslide}
   %
   \vspace*{\fill}

 %% \only<7>{
 %% \begin{center}
 %% \large{Thank you!}
 %% \end{center}
 %% }

\end{frame}%%%%%%%%%%%%%%%%%%%%%%%%%%%%%%%%%%%%%%%%%%%%%%%%%%%%%%%%%%%%%


\end{document}
